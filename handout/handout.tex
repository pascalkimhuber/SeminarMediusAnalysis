\documentclass[a4paper, 10pt]{article}

% Packages
% ---------------------------------------------------------------------------
\usepackage[utf8]{inputenc}
\usepackage{mathtools}
\usepackage{amssymb}
\usepackage{amsthm}
\usepackage{enumitem}
\usepackage{esint}
\usepackage{geometry}
\usepackage{fancyhdr}


% Commands
% ---------------------------------------------------------------------------
\newcommand{\real}{\mathbb{R}}
\newcommand{\norm}[1][\cdot]{\left\lVert#1\right\rVert}
\newcommand{\set}[1]{\left\{#1\right\}}
\newcommand{\dx}{\,dx}
\newcommand{\aop}[2]{a\left(#1, #2 \right)}
\newcommand{\bop}[2]{b\left(#1, #2 \right)}
\newcommand{\lprod}[2]{\left\langle#1, #2 \right\rangle_{L^2}}
\newcommand{\mesh}{\mathcal{T}_h}
\newcommand{\edges}{\mathcal{E}_h}
\newcommand{\nodes}{\mathcal{N}_h}
\newcommand{\crs}{CR_0^1}
\newcommand{\nci}{I_{NC}}
\newcommand{\lpo}{\Pi_0}

\DeclareMathOperator{\divop}{div}
\DeclareMathOperator{\diamop}{diam}
\DeclareMathOperator{\midop}{mid}
\DeclareMathOperator{\conv}{conv}
\DeclareMathOperator{\osc}{osc}


% Environments
% ---------------------------------------------------------------------------
\theoremstyle{definition}
\newtheorem{problem}{Problem}
\newtheorem{definition}{Definition}

\theoremstyle{plain}
\newtheorem{satz}{Satz}
\newtheorem{lemma}{Lemma}
\newtheorem*{satz*}{Satz}

% Margin
% ---------------------------------------------------------------------------
\geometry{a4paper, left=34mm, right=34mm, bottom=34mm, top=30mm}


% Header
% ---------------------------------------------------------------------------
\pagestyle{fancy} %eigener Seitenstil
\fancyhf{} %alle Kopf- und Fußzeilenfelder bereinigen
\fancyhead[L]{{\large \textbf{Handout: CRFEM für das Stokes
    Problem}}\\S5E1: Medius Analyse für nicht-konforme FEM} %Kopfzeile links
\fancyhead[C]{} %zentrierte Kopfzeile
\fancyhead[R]{Pascal Huber\\30.6.2014} %Kopfzeile rechts
\renewcommand{\headrulewidth}{0.4pt} %obere Trennlinie
\fancyfoot[C]{\thepage} %Seitennummer
\renewcommand{\footrulewidth}{0.0pt} %untere Trennlinie



% Document
% ---------------------------------------------------------------------------


\begin{document}
\section{Das Stokes Problem}
\label{sec:das-stokes-problem}

\subsubsection*{Notation}
\label{sec:notation}

\begin{itemize}
\item \(\Omega \subseteq \real^2\) beschränktes polygonales
  Lipschitz-Gebiet, \(f \in L^2(\Omega; \real^2)\) 
\item \(V\coloneqq H^1_0(\Omega; \real^2)\) mit Norm \(\norm_V
  \coloneqq \norm_{H^1(\Omega; \real^2)}\)\\
  \(Q\coloneqq L^2_0(\Omega) \coloneqq  \set{q \in L^2(\Omega):
    \int_\Omega q \dx = 0}\) mit Norm \(\norm \coloneqq \norm_{L^2(\Omega)}\)
\item \(a\colon V\times V \to \real\) Bilinearform definiert durch \((u,v) \mapsto \int_\Omega
  \nabla{u} : \nabla{v}\dx \) \\(mit \(A:B = \sum_{i,j = 1}^2 A_{ij}B_{ij}\) für \(A,B \in \real^{2\times2}\))
\item \(b\colon V\times Q \to \real\) Bilinearform definiert durch \((v,q) \mapsto \int_\Omega q
  \divop v\dx\)
\end{itemize}

\begin{problem}[Schwache Formulierung] \label{pro:1}
  Finde \((u,p)\in V\times Q\), sodass 
  \begin{equation}
    \label{eq:1}
    \begin{cases}
      \aop{u}{v} - \bop{v}{p} = \lprod{f}{v} & \text{für alle } v\in V, \\
      \qquad\qquad\; \bop{u}{q} = 0          & \text{für alle } q\in Q. 
    \end{cases}
  \end{equation}
\end{problem}

\begin{satz*}[Brezzis Splitting-Theorem] \label{thm:1}
  Ein Problem der Form (\ref{eq:1}) hat genau dann eine eindeutige
  Lösung \((u,p) \in V\times Q\), falls folgende Bedingungen erfüllt
  sind: 
  \begin{enumerate}[label=(\roman*)]
  \item Die Bilinearform \(a(\cdot, \cdot)\) ist elliptisch auf \(Z
    \coloneqq \set{v \in V\colon \forall q \in Q\colon \bop{v}{q} = 0}\).
  \item Die Bilinearform \(\bop{\cdot}{\cdot}\) erfüllt die
    \emph{inf-sup-Bedingung}: 
    \[
      \exists\, \beta > 0: \inf_{q\in Q} \sup_{v\in V} \frac{\bop{v}{q}} {\norm[v]_V\norm[q]} \geq \beta.
      \]
  \end{enumerate}
\end{satz*}



\section{CRFEM für das Stokes Problem}
\label{sec:crfem-fur-das}

\subsubsection*{Notation}
\label{sec:notation-1}

\begin{itemize}
\item \(\mesh\) quasi-uniforme Triangulierung von \(\Omega\), \(h \coloneqq \max\set{\diamop(T)\colon T \in \mesh}\)
\item \(\edges, (\edges^i/\edges^b)\) Menge aller (inneren/äußeren)
  Kanten von \(\mesh\)\\
  analog \(\nodes, (\nodes^i/\nodes^b)\) für die Knoten von \(\mesh\)
\item \(\midop(E)\) Mittelpunkt von \(E \in \edges\),
  \(\midop(\edges)\) Menge aller Kantenmittelpunkte
\item \(\nabla_h\) stückweiser Gradient: \((\nabla_h v)\vert_T = \nabla
  (v\vert_T)\) für \(T\in \mesh\) und geeignetes \(v\colon \Omega \to
  \real^2\), analog \(\divop_h\)
\end{itemize}

\subsubsection*{Crouzeix-Raviart Diskretisierung}
\label{sec:crouz-ravi-diskr}

\noindent Definiere die Räume
\begin{itemize}
\item \(P_k(\mesh) \coloneqq \set{v_h \in L^2(\Omega)\colon \forall T
    \in \mesh\colon v_h\vert_T \in P_k(T)}\),\\
  \(\crs(\mesh) \coloneqq \set{v_h \in P_1(\mesh)\colon v_h \text{
      stetig in } \midop(\edges^i), v_h(\midop(\edges^b)) = \set{0}}\),
\item \(V_h \coloneqq \left(\crs(\mesh)\right)^2\) mit Norm
  \(\norm[v]_h \coloneqq \norm[\nabla_h v]\) für \(v \in V + V_h\),\\
  \(Q_h \coloneqq P_0(\mesh) \cap L^2_0(\Omega)\) mit Norm
  \(\norm\), 
\end{itemize}

\noindent und die Bilinarformen \(a_h\colon V_h \times V_h \to \real\)
und \(b_h \colon V_h \times Q_h \to \real\) durch 
\[
    a_h(u_h, v_h) \coloneqq \int_\Omega \nabla_h u_h : \nabla_h v_h \dx
    \quad \text{ und } \quad b_h(v_h, q_h) \coloneqq \int_\Omega q_h\divop_h v_h \dx. 
\]

\begin{problem}[Diskretes Problem] \label{pro:2}
  Finde \((u_h, p_h) \in V_h \times Q_h\), sodass 
  \begin{equation}
    \label{eq:2}
    \begin{cases}
      a_h(u_h, v_h) - b_h(v_h, p_h) = \lprod{f}{v_h}   & \text{ für alle } v_h \in V_h, \\
      \qquad\qquad\quad\;\; b_h(u_h, q_h) = 0          & \text{ für alle } q_h \in Q_h. 
    \end{cases}
  \end{equation}
\end{problem}

\subsubsection*{Stabilität der CRFEM}
\label{sec:stabilitat-der-crfem}

\begin{definition}[Nicht-konformer Interpolationsoperator und
  \(L^2\)-Projektion] \label{def:2}
  Wir definieren 
  \begin{enumerate}[label=\emph{(\roman*)}]
  \item den \emph{nicht-konformen Interpolationsoperator} \(\nci\colon
    V \to V_h\), durch 
    \[
    (\nci v)(\midop(E)) \coloneqq \fint_E v \,ds \quad \text{für } E \in \edges
    \]
  \item die \emph{\(L^2\)-Projektion} durch \(\lpo \colon L^2(\Omega) \to
    P_0(\mesh)\),  
    \[
    (\lpo f)|_T \coloneqq \fint_T f \dx \quad \text{für } T\in \mesh. 
    \]
  \end{enumerate}
\end{definition}

\begin{lemma}[Eigenschaften von \(\nci\) und \(\lpo\)] \label{lem:1}
  \begin{enumerate}[label=(\roman*)]
  \item Für alle \(v\in V\) und \(T\in \mesh\) gilt \(\fint_T \nabla v
    \dx = \fint_T \nabla_h(\nci v) \dx\) und 
    \[
    \norm[v-\nci v]_h \leq \inf_{w_h \in V_h} \norm[v - w_h]_h.
    \]
  \item Für alle \(f\in L^2(\Omega)\) und \(T\in \mesh\) gilt
    \(\fint_T f \dx = \fint_T \lpo f \dx\) und 
    \[
    \norm[f-\lpo f] \leq \inf_{g\in P_0(\mesh)}\norm[f-g]. 
    \]
  \end{enumerate}
\end{lemma}

\begin{satz}[Stabilität von CR-NCFEM]\label{thm:3}
  Die Crouzeix-Raviart Diskretisierung des Stokes Problems \(V_h
  \times Q_h\) aus Problem \ref{pro:2} ist stabil, d.h. es gibt
  \(\beta > 0\) unabhängig von \(h\), so dass die \emph{diskrete
  inf-sup-Bedingung} erfüllt ist:
  \[
  \inf_{q_h \in Q_h} \sup_{v_h \in V_h} \frac{b_h(v_h, q_h)}{\norm[v_h]_h \norm[q_h]} \geq \beta. 
  \]
\end{satz}

\thispagestyle{plain}


\section{Fehlerabschätzung der CRFEM}
\label{sec:fehl-der-crfem}

\begin{definition}[Konforme Begleitabbildung \(J_k\)]\label{def:3}
  Wir definieren die \emph{konformen Begleitabbildungen} \(J_k \colon
  \crs(\mesh) \to P_k(\mesh)\cap C_0(\Omega)\) für
  \(k=1,2,3\) wie folgt für \(v_h \in \crs(\mesh)\): 
  \begin{enumerate}[label=\textit{(\roman*)}]
  \item \(J_1 v_h \coloneqq \sum_{z \in
      \nodes^i}\left(\left\lvert \mathcal{T}(z)\right\rvert^{-1}
      \sum_{T\in \mathcal{T}(z)} v_h\vert_T(z) \varphi_z \right)\)
  \item \(J_2 v_h \coloneqq J_1 v_h + \sum_{E\in\edges^i} 
    \left(\fint_E (v_h - J_1 v_h) \,ds\right)b_E\)
  \item \(J_3 v_h \coloneqq J_2 v_h + \sum_{T\in\mesh}\left(\fint_T(v_h
      - J_2 v_h)\dx \right)b_T\). 
  \end{enumerate}
  Dabei bezeichnet \(\varphi_z\) die konforme nodale Basisfunktion
  bzgl. \(z \in \nodes\), \(b_E \coloneqq 6\varphi_a\varphi_b\) die
  Bubble-Funktion bzgl. der Kante \(E \coloneqq \conv\{a,b\} \in
  \edges\) und \(b_T \coloneqq 60\varphi_a\varphi_b\varphi_c\) die
  Bubble-Funktion für das Element \(T \coloneqq \conv\{a,b,c\} \in
  \mesh\). 
\end{definition}

\vspace{3mm}

\begin{lemma}[Eigenschaften von \(J_3\)]\label{lem:2}
  Die konforme Begleitabbildung \(J_3\colon \crs(\mesh) \to
  P_3(\mesh)\cap C_0(\Omega)\) hat folgende Eigenschaften: 
  \begin{enumerate}[label=(\roman*)]
  \item \(\int_T \nabla_h(v_h - J_3 v_h)\dx = 0\) und \(\int_T
    v_h - J_3 v_h \dx = 0 \) für \(v_h \in
    \crs(\mesh) \text{ und } T\in \mesh\),
  \item \(\norm[\nabla_h (v_h - J_3 v_h)] \lesssim \norm[h^{-1}(v_h -
      J_3 v_h)] \lesssim \norm[\nabla_h v_h] \quad\) für alle \(v_h \in
    \crs(\mesh)\).
  \end{enumerate}
\end{lemma}

\vspace{3mm}

\begin{satz}[Best-Approximation der CRFEM]\label{thm:4}
  Sei \((u,p) \in V\times Q\) die Lösung von (\ref{eq:1}) und
  \((u_h,p_h) \in  V_h\times Q_h\) die Lösung von (\ref{eq:2}). Dann
  gilt: 
  \begin{equation}
    \label{eq:3}
    \norm[u - u_h]_h + \norm[p - p_h] \lesssim \inf_{v_h \in V_h} \norm[u - v_h]_h + \inf_{q_h \in Q_h} \norm[p - q_h] + \osc(f,h), 
  \end{equation}
  wobei die Oszillation von \(f\) durch  \(\osc(f, h) \coloneqq
  \norm[h(f - \lpo f)]\) definiert ist. 
\end{satz}

\end{document}
