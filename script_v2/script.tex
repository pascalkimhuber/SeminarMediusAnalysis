\documentclass[a4paper]{scrartcl}

% --------------------------------------- Packages
\usepackage[utf8]{inputenc}
\usepackage{mathtools}
\usepackage{amssymb}
\usepackage{amsthm}
\usepackage{enumitem}
\usepackage{bbold}
\usepackage{draftwatermark}
\usepackage{xcolor}
\usepackage{esint}

% --------------------------------------- Definitions

\newcommand{\real}{\mathbb{R}}

\newcommand{\ubold}{\boldsymbol{u}}
\newcommand{\fbold}{\boldsymbol{f}}

\newcommand{\Hzero}{H_0^1}
\newcommand{\Ltwo}{L^2}
\newcommand{\crfem}{\operatorname{CR}_0^1}
\newcommand{\czero}{\operatorname{C}_0}
\newcommand{\J}[1]{J_{#1}}


\newcommand{\mesh}{\mathcal{T}_h}
\newcommand{\edges}{\mathcal{E}}
\newcommand{\nodes}{\mathcal{N}}

\newcommand{\dx}{\,dx}
\newcommand{\vnorm}[1]{\left\lVert#1\right\rVert_V}
\newcommand{\norm}[1]{\left\lVert#1\right\rVert}
\newcommand{\hnorm}[1]{\left\lVert#1\right\rVert_h}
\newcommand{\scpr}[2]{\left\langle#1,#2\right\rangle_{L^2}}
\newcommand{\set}[1]{\left\{#1\right\}}

\newcommand{\black}{\color{black}}
\newcommand{\blue}{\color{blue}}
\newcommand{\red}{\color{red}}

\DeclareMathOperator{\divOp}{div}
\DeclareMathOperator{\divh}{div_h}
\DeclareMathOperator{\nablah}{\nabla_{\textit{h}}}
\DeclareMathOperator{\diam}{diam}
\DeclareMathOperator{\midOp}{mid}
\DeclareMathOperator{\intOp}{I_{NC}}
\DeclareMathOperator{\LtwoOp}{\Pi_0}
\DeclareMathOperator{\conv}{conv}
\DeclareMathOperator{\osc}{osc}


% --------------------------------------- Environments
\theoremstyle{plain}
\newtheorem{theorem}{Satz}
\newtheorem{lemma}{Lemma}
\theoremstyle{definition}
\newtheorem{definition}{Definition}
\newtheorem{modellproblem}{Modellproblem}
\theoremstyle{remark}
\newtheorem*{remark*}{Bemerkung}

% --------------------------------------- Some settings
\SetWatermarkScale{4}

% --------------------------------------- Document
\begin{document}
\title{\textsc{Eine CRFEM für das Stokes Problem}}
\author{Pascal Huber}
\date{30. Juni 2014}
\maketitle

\section{Motivation und Einleitung}
\label{sec:motiv-und-einl}

\blue 
Nachdem wir in den letzten Vorträgen die nicht-konforme
Crouzeix-Raviart Diskretisierung für die Poissongleichung näher
beleuchtet haben, wollen wir in diesem Vortrag das Crouzeix-Raviart
Element zur numerischen Lösung der Stokes Gleichung
\black
\begin{equation}
  \label{eq:1}
  -\Delta u + \nabla p = f \text{ in } \Omega \quad \text{ und }
  \quad \divOp u = 0 \text{ in } \Omega
\end{equation}
\blue
verwenden. Im Gegensatz zur Poissongleichung ist für die Stokes
Gleichung eine Diskretisierung mit Hilfe von konformen \(P_1\)
Elementen nicht möglich, sodass eine einfache nicht-konforme Methode
durchaus ihre Daseinsberechtigung hat.  \\

\noindent Im Folgenden wollen wir zunächst die schwache Formulierung
des Stokes Problems, sowie die Bedingungen zur eindeutigen Existenz von
Lösungen wiederholen. \\
Anschließend definieren wir eine Finite Elemente
Methode (FEM) für das Problem mit Hilfe von nicht-konformen
Crouzeix-Raviart Elementen und zeigen, dass diese so definierte
Diskretisierung stabil ist, d.h. eine bestimmte \emph{inf-sup-Bedingung}
erfüllt ist. \\
Dies ermöglicht uns dann auch im letzten Teil eine
Fehlerabschätzung zur Bestapproximation dieser FEM mittels \emph{Medius
Analyse} durchzuführen. 



\section{Das Stokes Problem}
\label{sec:das-stokes-problem}

\paragraph{Schwache Formulierung des Stokes Problems}
\label{sec:schw-form-des}

Das Stokes Problem beschreibt den Druck und das Geschwindigkeitsfeld
von  inkompressiblen Flüssigkeiten mit hoher Viskosität (und ist damit
eine linearisierte Vereinfachung der Navier-Stokes-Gleichungen).\\ 

\noindent Die schwache Formulierung des Stokes Problem führt
mathematisch zu einem Sattelpunktproblem. Wir betrachten das folgende Modellproblem:

\black
\begin{modellproblem}[Schwache Formulierung] \label{problem:1}
  Sei \(\Omega \subset \real^2\) ein polygonales beschränktes
  Lipschitz-Gebiet und  
  \[V \coloneqq \Hzero(\Omega; \real^2) \text{ und } Q \coloneqq
  \Ltwo_0(\Omega) \coloneqq \{q \in \Ltwo(\Omega): \int_\Omega q(x)
  \dx = 0\}.\]
  Wir definieren die Bilinearformen \(a \colon V\times V \rightarrow \real\)
  und \(b \colon V\times Q \rightarrow \real\) durch
  \begin{equation}
    \label{eq:2}
    a(u,v) \coloneqq \int_\Omega \nabla u : \nabla v \dx 
    \text{ und }
    b(v,q) \coloneqq \int_\Omega \divOp v \cdot q \dx \nonumber
  \end{equation}
  und die Normen \(\vnorm{\cdot} \coloneqq \norm{\cdot}_{H^1(\Omega;
    \real^2)}\) und \(\norm{\cdot} \coloneqq \norm{\cdot}_{\Ltwo(\Omega)}\).\\
  Für \(f\in \Ltwo(\Omega; \real^2)\) finde \((u,p) \in V\times Q\), sodass folgende Gleichungen erfüllt sind: 
  \begin{equation}
    \label{eq:3}
    \begin{cases}
      a(u,v) - b(v,p) = \langle f, v \rangle_{\Ltwo(\Omega;\real^2)} &
      \text{ für alle } v \in V \\
      b(u,q) = 0 & \text{ für alle } q \in Q.
    \end{cases}
  \end{equation}
\end{modellproblem}

\blue
\noindent Hierbei bezeichnet \(u \in V\) das Geschwindigkeitsfeld der
Flüssigkeit und \(p\in Q\) deren Druck. In der Definition von
\(a(\cdot, \cdot)\) bezeichnet \(A:B\) das Skalarprodukt für Matrizen
aus \(\real^{2\times 2}\): 
\black
\[ A:B \coloneqq \sum_{i,j = 1}^2 A_{i,j}B_{i,j}.\]



\paragraph{Existenz und Eindeutigkeit von Lösungen des Stokes Problems}
\label{sec:exist-und-eind}

\blue
\noindent Bevor wir uns der Diskretisierung des Problems zuwenden, bleibt
noch die Frage nach Existenz und Eindeutigkeit der Lösung von
Problem \ref{problem:1} zu klären. \\

\noindent Da es sich hier um ein ``gemischtes Problem'' mit zwei
Unbekannten handelt, ist der gewöhnliche Ansatz für den Existenzbeweis
mit dem Lemma von Lax-Milgram nicht möglich. Stattdessen verwendet man das
\emph{Splitting-Theorem von Brezzi}, das  die Existenz und Eindeutigkeit von
Lösungen für allgemeine Sattelpunktprobleme liefert. \\
\black
Dazu muss gezeigt werden, dass zum einen die Bilinearform \(a(\cdot,
\cdot)\) auf dem Unterraum 
\begin{equation}
  \label{eq:29}
  Z \coloneqq \set{v \in V: \forall q \in Q: b(v,q) = 0}
\end{equation}
elliptisch ist und dass \(b(\cdot, \cdot)\) die folgende
inf-sup-Bedingung erfüllt: 
\begin{equation}
  \label{eq:30}
  \exists \beta > 0:\qquad \inf_{q\in Q} \sup_{v \in V} \frac{b(v,q)}{\vnorm{v}\norm{q}} \geq \beta. 
\end{equation}
Im Fall der Stokes Gleichung sind beide Bedingungen erfüllt, was die
eindeutige Existenz der Lösung \((u,p) \in V\times Q\) liefert. 



\section{CR-Diskretisierung und Stabilität}
\label{sec:cr-diskr-und}

\subsection{Motivation}
\label{sec:motivation}

\blue
Die Schwierigkeiten bei der Wahl geeigneter FE-Räume für das Stokes
Problem bestehen nicht nur darin gute Konvergenzraten der diskreten
Lösung zu erreichen, sondern auch schon in der Frage nach der
eindeutigen Existenz dieser Lösungen. \\

\noindent Wie schon bei der Untersuchung der Existenz und
Eindeutigkeit von Lösungen für das kontinuierliche Stokes Problem müssen
bei der Diskretisierung zwei Unbekannte bestimmt werden. Dies
erfordert zum einen die Wahl von gemischen Finiten Elementen und zum
anderen muss für diese FE-Räume wieder eine (diskrete)
inf-sup-Bedingung erfüllt sein, um die eindeutige Existenz der
Approximation sicherzustellen. Man spricht dabei auch von
\emph{inf-sup-Stabilität}. \\ 

\noindent Es stellt sich zum Beispiel heraus, dass die Diskretisierung des
Stokes Problems mit Hilfe von konformen \(P_1\)-Elementen für die
Geschwindigkeit und \(P_0\)-Elementen für den Druck nicht
inf-sup-stabil sind. Stattdessen erhält man durch das Hinzufügen von
sog. \emph{bubble}-Funktionen zum Geschwindigkeitsraum das strukturell
einfachste konforme FE-Paar. \\ 

\black
\noindent Aus diesem Grund ist die Frage nach einfachen
nicht-konformen FEM für das Stokes Problem durchaus berechtigt. Einen
Hinweis darauf, dass Crouzeix-Raviart-Elemente (CRFEM) zusammen mit
stückweise konstanten Druckfunktionen das Stokes Problem gut
diskretisieren gibt die Tatsache, dass das diskrete Analogon zur Menge
\(Z\) in (\ref{eq:29}) 
\begin{equation}
  \label{eq:31}
  Z_h \coloneqq \set{v_h \in V_h\colon \forall q_h \in Q_h\colon b_h(v_h, q_h) = 0} 
\end{equation}
im Fall der CRFEM nicht trivial ist (im Gegensatz zur konformen
\(P_1\)-\(P_0\)-Diskretisierung). Die Lösungsräume \(V_h\) und \(Q_h\)
scheinen also für das Crouzeix-Raviart Element gut ausgewogen zu sein. 


\subsection{Beschreibung der Diskretisierung}
\label{sec:beschr-der-diskr}

Zur Diskretisierung von Modellproblem \ref{problem:1} verwenden wir
eine quasi-uniforme Triangulierung \(\mesh\) des Gebiets \(\Omega\)
und führen die folgenden Bezeichungen ein: 

\begin{itemize}
\item \(h \coloneqq \max\{h_T\colon T \in \mesh\}\) mit \(h_T \coloneqq \diam(T)\) für \(T\in \mesh\)
\item \(\edges\) und \(\nodes\):  Menge aller Kanten
  bzw. Knoten von \(\mesh\)
\item \(\midOp(E)\): Kantenmittelpunkt von \(E \in \edges\), analog \(\midOp(\edges)\)
\item \(\nablah\) und \(\divh\): Elementweiser Gradient bzw. Divergenz
\end{itemize}

\noindent Desweiteren definieren wir die Funktionenräume
\begin{align}
  \label{al:2}
P_k(\mesh) &\coloneqq \set{v_h \in \Ltwo(\Omega)\colon \forall T \in
  \mesh\colon v_h|_T \in P_k(T)} \\
  \crfem(\mesh) &\coloneqq \set{v_h \in P_1(\mesh)\colon v_h \text{ stetig in } \midOp(\edges), v_h(\midOp(E)) = 0 \text{ für } E \in \partial\Omega\cap\edges}
\end{align}
und die Normen
  \begin{equation}
    \label{eq:10}
    \norm{f} \coloneqq \norm{f}_{\Ltwo(\Omega)} \quad \text{ und } \quad
    \hnorm{v} \coloneqq \norm{\nablah v}  
  \end{equation}
für \(f\in \Ltwo(\Omega)\) und \(v \in \crfem(\Omega) +
\Hzero(\Omega)\). (Analog für vektorwertige Funktionen.)


\noindent Mit diesen Bezeichnungen können wir nun das diskrete
Modellproblem zu (\ref{eq:3}) formulieren: 

\begin{modellproblem}[CR-Diskretisierung] \label{problem:3}
  Seien \(Q_h \coloneqq \Ltwo_0(\Omega) \cap
  P_0(\mesh)\) und \(V_h \coloneqq \crfem(\mesh) \times \crfem(\mesh)\).
  Weiter seien die diskreten Bilinearformen \(a_h\colon V_h \times V_h
  \rightarrow \real\) und \(b_h\colon V_h \times Q_h \rightarrow
  \real\) definiert durch
  \begin{equation}
    \label{eq:11}
    a_h(u_h, v_h) \coloneqq \int_\Omega \nablah u : \nablah v \dx
    \quad \text{ und } \quad b_h(v_h, q_h) \coloneqq \int_\Omega q_h\divh v_h \dx. 
  \end{equation}
  Für gegebenes \(f \in \Ltwo(\Omega; \real^2)\) finde \((u_h, p_h)
  \in V_h \times Q_h\), sodass 
  \begin{equation}
    \label{eq:12}
    \begin{cases}
      a_h(u_h, v_h) - b_h(v_h, p_h) = \langle f, v_h \rangle_{\Ltwo(\Omega; \real^2)} & \text{ für alle } v_h \in V_h \\
      b_h(u_h, q_h) = 0 & \text{ für alle } q_h \in Q_h. 
    \end{cases}
  \end{equation}
\end{modellproblem}

\noindent Die Methode ist offensichtlich nicht-konform, da offensichtlich \(V_h \not\subseteq V\).



\subsection{Stabilität der CR-NCFEM}
\label{sec:stabilitat-der-cr}

Wie bereits erwähnt, zeigt man die Existenz und Eindeutigkeit der
diskreten Lösung \((u_h, p_h)\) in Modellproblem \ref{problem:3}
wieder mit Hilfe des Brezzi-Splitting Theorems.\\ 

\noindent Die Definition der Normen \(\norm{\cdot}_h\) und
\(\hnorm{\cdot}\) stellt die Stetigkeit der Bilienarformen, sowie die
Elliptizität von \(a(\cdot, \cdot)\) sicher. Das eigentliche Kriterium
für die Existenz und Eindeutigkeit ist also wieder die
inf-sup-Bedingung. 

\begin{definition}[Stabilität von FEM] \label{def:1}
  Eine FE-Diskretisierung \(V_h\times Q_h\) für das Modellproblem
  \ref{problem:1} heißt \emph{(inf-sup-)stabil}, falls \(\beta > 0\)
  unabhänging von \(h\) exisitiert, sodass  
  \begin{equation}
    \label{eq:14}
    \inf_{q_h \in Q_h} \sup_{v_h \in V_h} \frac{b_h(v_h, q_h)}{\hnorm{v_h} \norm{q_h}} \geq \beta. 
  \end{equation}
  \end{definition}

\noindent Die inf-sup-Stabilität ist darüber hinaus auch für die
Stabilität der numerischen Berechnung wichtig. 

\begin{theorem}[Stabilität von CR-NCFEM]\label{thm:3}
  Die Crouzeix-Raviart Diskretisierung des Stokes Problems \(V_h \times Q_h =
  \crfem(\mesh)^2 \times (\Ltwo_0(\Omega)\cap P_0(\mesh))\) aus
  Modellproblem \ref{problem:3}  ist stabil. 
\end{theorem}

\noindent Für den Beweis der Stabilität definieren wir einen
nicht-konformen Interpolationsoperator, der auch noch später für die
Medius Analyse eine Rolle spielen wird.  

\begin{definition}[Nicht-konformer Interpolationsoperator und
  \(\Ltwo\)-Projektion] \label{def:2}
  Wir definieren: 
  \begin{enumerate}[label=\emph{(\roman*)}]
  \item \( \intOp\colon \Hzero(\Omega) \rightarrow \crfem(\mesh), \quad
      (\intOp v)(\midOp(E)) \coloneqq \fint_E v \,ds \qquad \text{für } E \in \edges.\)  
    %\end{equation}
  \item \(\LtwoOp \colon \Ltwo(\Omega) \rightarrow P_0(\mesh), \quad 
      (\LtwoOp f)|_T \coloneqq \fint_T f \dx \qquad \text{für } T\in \mesh.\) 
    \end{enumerate}
\end{definition}

\noindent Die Definition von \(\intOp\) und \(\LtwoOp\) können auf natürliche
Weise auf vektorwertige Funktionen erweitert werden. \\ 

\noindent Die Motivation für Definition \ref{def:2} von \(\intOp\) und
\(\LtwoOp\) besteht darin, dass sie jeweils eine Bestapproximation
ihres Arguments darstellen. 

\begin{lemma}[Eigenschaften von \(\intOp\) und \(\LtwoOp\)] \label{lem:1}
  \begin{enumerate}[label=(\roman*)]
  \item Für alle \(v\in V\) und \(T\in \mesh\) gilt \(\fint_T \nabla v
    \dx = \fint_T \nabla(\intOp v) \dx\) und 
    \begin{equation}
      \label{eq:16}
      \norm{\nablah(v-\intOp v)} \leq \inf_{w\in P_0{\mesh}^2} \norm{\nablah v - w}.
    \end{equation}
  \item Für alle \(f\in \Ltwo(\Omega\) und \(T\in \mesh\) gilt
    \(\fint_T f \dx = \fint_T \LtwoOp f \dx\) und 
    \begin{equation}
      \label{eq:17}
      \norm{f-\LtwoOp f} \leq \inf_{g\in P_0(\mesh)}\norm{f-g}. 
    \end{equation}
  \end{enumerate}
\end{lemma}

% \begin{proof}[Beweis.]
%   \begin{enumerate}[label=\emph{(\roman*)}]
%   \item Mit partieller Integration gilt und der Definition von
%     \(\intOp\) folgt:  
%     \[\int_T \nabla(v - \intOp v)\dx = \sum_{E\in\edges(T)} \int_E (v - \intOp v)\otimes \nu_T \,ds = 0\]
%     Für (\ref{eq:16}) rechnen wir: 
%     \(v \in V\), \(w\in P_0(\mesh)^2\). Dann gilt: 
%     \begin{align}
%       \norm{\nablah v - w}^2 =& \norm{\nablah (v - \intOp v)}^2 + \norm{\nablah \intOp v - w}^2 \\ \nonumber 
%       & + \int_\Omega \left(\nablah(v - \intOp v)\right) \cdot \left(\nablah \intOp v - w\right) \dx. \nonumber
%     \end{align}
%     Der letzte Summand verschwindet wegen der Mittelwerteigenschaft
%     von \(\intOp\). Der Beweis für \(\LtwoOp\) geht analog. 
% \end{enumerate}
% \end{proof}

\noindent Wir können nun den nicht-konformen Interpolationsoperator \(\intOp\)
verwenden, um die Stabilität der Crouzeix-Raviart FE-Räume zu zeigen. 

\begin{proof}[Beweis von Satz \ref{thm:3}.]
  Für den Beweis zeigen, wir dass der nicht-konforme
  Interpolationsoperator ein \emph{Fortin-Operator} ist, d.h. wir
  zeigen: 
 \begin{align}     
   &(i) \quad \exists c > 0 \text{ (unabhängig von } h\text{)} \colon
   \hnorm{\intOp v} \leq c \vnorm{v} \text{ für alle } v \in V \nonumber\\ 
   &(ii) \quad \forall v \in V, \forall q_h \in Q_h\colon b_h(\intOp v, q_h) = b(v, q_h) \nonumber
   \label{align:2} 
  \end{align}
  Da \(\nablah (\intOp v)\) stückweise konstant ist, folgt mit
  (\ref{eq:16})
  \begin{equation}
    \label{eq:18}
    \int_\Omega \nablah (\intOp v) : \nablah (\intOp v) \dx = 
    \int_\Omega \nabla v : \nablah (\intOp v) \dx \leq \norm{\nabla v}\hnorm{\intOp v}
  \end{equation}
  und damit \emph{(i)} mit \(c=1\). Auch \emph{(ii)} folgt aus der
  Mittelwerteigenschaft, da \(q_h \in Q_h\) stückweise konstant ist. \\ 
  Sei nun \(q_h \in Q_h\) beliebig. Dann erhält man aus der
  inf-sup-Bedingung von \(b(\cdot, \cdot)\) und \emph{(i), (ii)}: 
  \begin{align}
    \beta \norm{q_h} 
    &\leq \sup_{v\in V} \frac{b(v,q_h)}{\vnorm{v}}
    \overset{\emph(ii)}{=} \sup_{v\in V} \frac{b_h(\intOp v,q_h)}{\vnorm{v}}
    \overset{\emph(i)}{\leq} \sup_{v\in V} \frac{b_h(\intOp v,q_h)}{\hnorm{\intOp v}} \nonumber 
    \leq \sup_{v_h\in V_h} \frac{b_h(v_h,q_h)}{\hnorm{v_h}}. \nonumber
  \end{align}
\end{proof}

\begin{remark*}
  Wir haben gesehen, dass die Ungleichung in \emph{(i)} mit der
  bestmöglichen Konstante \(c=1\) für den nicht-konformen
  Interpolationsoperator \(\intOp\) gilt. 
\end{remark*}



\section{Medius Analyse der CRFEM für das Stokes Problem}
\label{sec:medius-analyse-der}

\blue
In den vorangegangen Abschnitten haben wir gesehen, dass die
Crouzeix-Raviart Diskretisierung eine relativ einfache FEM für das
Stokes Problem darstellt, die zudem sehr gute Stabilitätseigenschaften
aufweist. \\
Zum Abschluss des Vortrags wollen wir für die Fehleranalyse ein
Resultat zur Best-Approximation der Crouzeix-Raviart Diskretisierung
angeben und mit Hilfe einer Medius Analyse beweisen. 

\black
\begin{theorem}[Best-Approximation der CRFEM]\label{thm:4}
  Sei \((u,p) \in V\times Q\) die Lösung von (\ref{eq:3}) und
  \((u_h,p_h) \in  V_h\times Q_h\) die Lösung von (\ref{eq:12}). Dann
  gilt: 
  \begin{equation}
    \label{eq:19}
    \hnorm{u - u_h} + \norm{p - p_h} \lesssim \inf_{v_h \in V_h} \hnorm{u - v_h} + \inf_{q_h \in Q_h} \norm{p - q_h} + \osc(f,h), 
  \end{equation}
  wobei die Oszillation von \(f\) durch  \(\osc(f, h) \coloneqq
  \norm{h(f - \LtwoOp f)}\) definiert ist. 
\end{theorem}

\begin{remark*}
  Man beachte, dass für dieses Resultat keine weiteren
  Regularitätsannahmen an die Lösung \((u,p)\) gemacht werden muss. 
\end{remark*}

\subsection{Konforme Begleitabbildungen}
\label{sec:konf-begl}

\blue
Für den Beweis von Satz \ref{thm:4} verwenden wir wie schon im
letzten Vortag sogenannte \emph{konforme Begleitabbildungen} der Form 
\[ J_k \colon \crfem(\mesh) \rightarrow
P_k(\mesh)\cap\czero(\Omega).\] 
Dadurch können wir einen Zusammenhang zwischen dem nicht-konformen
Crouzeix-Raviart FE-Raum und konformen Räumen herstellen. \\

\noindent Wir definieren eine konforme Begleitabbildung von der
Ordnung \(k = 3\) wie folgt: 

\black
\begin{definition}[Konforme Begleitabbildung \(J_3\)]\label{def:3}
  Wir definieren wir die konformen Begleitabbildungen \(J_k \colon
  \crfem(\mesh) \rightarrow P_k(\mesh)\cap\czero(\Omega)\) für
  \(k=1,2,3\) wie folgt für \(v_h \in \crfem(\mesh)\): 
  \begin{enumerate}[label=\textit{(\roman*)}]
  \item \(J_1 v_h \coloneqq \sum_{z \in
      \nodes(\Omega)}\left(\left\lvert \mathcal{T}(z)\right\rvert^{-1}
      \sum_{T\in \mathcal{T}(z)} v_h\vert_T(z) \varphi_z \right)\)
  \item \(J_2 v_h \coloneqq J_1 v_h + \sum_{E\in\edges(\Omega)} 
    \left(\fint_E (v_h - J_1 v_h) \,ds\right)b_E\)
  \item \(J_3 v_h \coloneqq J_2 v_h + \sum_{T\in\mesh}\left(\fint_T(v_h
      - J_2 v_h)\dx \right)b_T\). 
  \end{enumerate}
  Dabei bezeichnen \(\nodes(\Omega)\) und \(\edges(\Omega)\) die Menge
  allere inneren Knoten bzw. Kanten und weiter \(\varphi_z\) die
  konforme nodale Basisfunktion bzgl. \(z \in \nodes\), \(b_E
  \coloneqq 6\varphi_a\varphi_b\) die Bubble-Funktion bzgl. der Kante
  \(E \coloneqq \conv\{a,b\} \in \edges\) und \(b_T \coloneqq
  60\varphi_a\varphi_b\varphi_c\) die Bubble-Funktion für das Element
  \(T \coloneqq \conv\{a,b,c\} \in \mesh\).
\end{definition}

\begin{remark*}
  Auf \(V_h\) ist \(J_k\) komponentenweise definiert.
\end{remark*}

\blue
\noindent Die Motivation für die Definition von \(\J3\) ist im
Wesentlichen durch das folgende Lemma gegeben. Es besagt, dass die
Begleitabbildung gewisse Mittelwerteigenschaften erfüllen, was wir im
Hauptbeweis mehrmals ausnutzen werden. 

\black
\begin{lemma}[Eigenschaften von \(J_3\)]\label{lem:2}
  Die konforme Begleitabbildung \(\J3\colon \crfem(\mesh) \rightarrow
  P_3(\mesh)\cap\czero(\Omega)\) hat folgende Eigenschaften: 
  \begin{enumerate}[label=(\roman*)]
  \item \(\int_T \nablah(v_h - \J3 v_h)\dx = 0\) und \(\int_T
    v_h - \J3 v_h \dx = 0 \) für \(v_h \in
    \crfem(\mesh) \text{ und } T\in \mesh\),
  \item \(\norm{\nablah (v_h - \J3 v_h)} \lesssim \norm{h^{-1}(v_h -
      \J3 v_h)} \lesssim \norm{\nablah v_h} \quad\) für alle \(v_h \in
    \crfem(\mesh)\).
  \end{enumerate}
\end{lemma}

{\red
Übersicht zu den verwendeten Relationen! 
}

\subsection{Beweis der Best-Approximation}
\label{sec:beweis-der-best}

\blue
Mit Hilfe von Lemma \ref{lem:2} sind wir nun in der Lage Satz
\ref{thm:4} in zwei Schritten zu beweisen: 

\begin{enumerate}
\item Für die Abschätzung von \(\hnorm{u - u_h}\) nutzen wir die
  Eigenschaften von \(\J3\) aus. 
\item Die Abschätzung des Druckterms \(\norm{p - p_h}\) leitet sich
  Wesentlichen aus der Stabilität des Crouzeix-Raviart Elements her. 
\end{enumerate}

\black
\begin{proof}[Beweis von Satz \ref{thm:4}.]
  \textit{\\Teil 1: Abschätzung für die Geschwindigkeit:}\\
  Wegen Lemma \ref{lem:1} genügt es
  \begin{equation}
    \label{eq:21}
    \hnorm{u - u_h} \lesssim \hnorm{u - \intOp u} + \norm{p - \LtwoOp p} + \osc(f, h)
  \end{equation}
  zu zeigen. Aus der Definition von \(a_h(\cdot,
  \cdot)\) folgt
  \begin{equation}
    \label{eq:20}
    \hnorm{u - u_h}^2 = a_h(u - u_h, u - \intOp u) + a_h(u - u_h, \intOp u - u_h). 
  \end{equation}
  Wir schätzen den zweiten Term in (\ref{eq:20}) ab. Dazu setzen wir \(w_h
  \coloneqq \intOp u - u_h \in V_h\). Aus \(\divOp u = 0\) und \(\divh
  u_h = 0\) erhält man mit der Mittelwerteigenschaft von \(\intOp\),
  dass auch \(\divh w_h = 0\) punktweise gilt. Damit folgt
  \begin{align}
    a_h(u &- u_h, w_h) \nonumber \\
    &= a_h(u,  w_h - \J3 w_h) + a(u, \J3 w_h) - a_h(u_h, w_h) \nonumber \\
    &= a_h(u,  w_h - \J3 w_h) + b(\J3 w_h, p) + \scpr{f}{\J3 w_h} \nonumber 
    - b_h(w_h, p_h) - \scpr{f}{w_h}. 
  \end{align}
  Wegen der Mittelwerteigenschaft von \(\J3\) gilt für alle \(g \in
  P_0(\mesh)^{2\times 2}\)
  \begin{equation}
    \label{eq:22}
    \int_\Omega g : \nablah (w_h - \J3 w_h) \dx = 0. 
  \end{equation}
  Mit \(\LtwoOp \nabla u = \nablah \intOp u\) erhält man deshalb durch
  Anwendung der Cauchy-Schwarz Ungleichung
  \begin{align} \label{al:1}
    a_h(u &- u_h, w_h) \nonumber \\
    =& a_h(u - \intOp u, w_h - \J3 w_h) + b(\J3 w_h, p - \LtwoOp p) + \scpr{f - \LtwoOp f}{\J3 w_h - w_h} \nonumber \\
    \leq& \hnorm{u - \intOp u}\hnorm{w_h - \J3 w_h} +\norm{p - \LtwoOp p}\hnorm{\J3 w_h} \nonumber \\
    & + \norm{h(f - \LtwoOp f)}\norm{h^{-1}(\J3 w_h - w_h)} \nonumber \\
    \overset{(ii)}{\lesssim}& \left(\hnorm{u - \intOp u} + \norm{p + \LtwoOp p} + \osc(f, h)\right) \hnorm{w_h}. 
  \end{align}
  Aus der Mittelwerteigenschaft von \(\intOp\) erhält man 
  \begin{equation}
    \label{eq:23}
    \hnorm{w_h} = (a_h(\intOp u - u_h, w_h))^{1/2} = (a_h( u - u_h, w_h))^{1/2}.
  \end{equation}
  Eingesetzt in (\ref{al:1}) ergibt das schließlich für den zweiten
  Summanden aus (\ref{eq:20})
  \begin{equation}
    \label{eq:24}
    a_h(u - u_h, \intOp u - u_h) \lesssim \hnorm{u - \intOp u} + \norm{p - \LtwoOp p} + \osc(f, h). 
  \end{equation}
  Für den ersten Summanden folgt mit der Youngschen Ungleichung
  \begin{equation}
    \label{eq:25}
    a_h(u - u_h, u - \intOp u) \leq \frac{1}{2}\left(\hnorm{u - u_h}^2 + \hnorm{u - \intOp u}^2 \right), 
  \end{equation}
  woraus schließlich zusammen mit (\ref{eq:24}) die Abschätzung
  (\ref{eq:21}) folgt. 

  \textit{\\Teil 2: Abschätzung für den Druck:}\\
  Es genügt wieder nur
  \begin{equation}
    \label{eq:26}
    \norm{p - p_h} \lesssim \hnorm{u_h - u} + \norm{p - \LtwoOp p} + \osc(f,h)
  \end{equation}
  zu zeigen. Dazu verwenden wir die inf-sup-Stabilität des
  Elementpaares \(V_h \times Q_h\). Da \((p_h - \LtwoOp p) \in Q_h\), gibt es ein Element
  \(v_h \in V_h\) mit \(\hnorm{v_h} = 1\), so dass 
  \begin{equation}
    \label{eq:27}
    \norm{p_h - \LtwoOp p} \lesssim b_h(v_h, p_h - \LtwoOp p). 
  \end{equation}
  Da \(p_h\) die Lösung des diskreten Problems ist gilt mit \(\LtwoOp
  v_h = \LtwoOp \J3 v_h\)
  \begin{equation}
    \label{eq:28}
    \norm{p_h - \LtwoOp p} \lesssim a_h(u_h, v_h) - \scpr{f}{v_h} - b_h(\J3 v_h,\LtwoOp p). 
  \end{equation}
  Durch Addition von \(0 = b(\J3 v_h, p) - a(u, \J3 v_h) +
  \scpr{f}{\J3 v_h}\) erhält man unter Ausnutzung der
  Mittelwerteigenschaft von \(\J3\): 
  \begin{align}
    &\norm{p_h - \LtwoOp p} \\
    &\qquad\lesssim a_h(u_h, v_h) - a(u, \J3 v_h) + b(\J3 v_h, p - \LtwoOp p) + \scpr{f}{\J3 v_h - v_h} \nonumber \\
    &\qquad= a_h(u_h - u, v_h) - a(u, v_h - \J3 v_h) + b(\J3 v_h, p - \LtwoOp p) + \scpr{f}{\J3 v_h - v_h} \nonumber \\
    &\qquad= a_h(u_h - u, v_h) - a(u - \intOp u, v_h - \J3 v_h) \nonumber \\
    &\qquad \quad+ b(\J3 v_h, p - \LtwoOp p) + \scpr{f - \LtwoOp f}{\J3 v_h - v_h}. \nonumber 
  \end{align}
  Mit der Cauchy-Schwarz Ungleichung und Lemma \ref{lem:2}, sowie
  \(\hnorm{u - \intOp u} \leq \hnorm{u - u_h}\) (siehe Lemma
  \ref{lem:1}) erhält man schließlich wegen \(\hnorm{v_h} = 1\)
  \begin{align}
    \norm{p_h - \LtwoOp p} \lesssim
    & \hnorm{u_h - u}\hnorm{v_h} + \hnorm{u - \intOp u}\hnorm{v_h - \J3 v_h} \nonumber \\
    &+ \norm{p - \LtwoOp p}\hnorm{\J3 v_h} \norm{h(f - \LtwoOp f)}\hnorm{h^{-1}(\J3 v_h - v_h)} \nonumber \\
    \lesssim& \left(\hnorm{u_h - u} + \norm{p - \LtwoOp p} + \osc(f,h)\right) \hnorm{v_h} \nonumber \\
    \lesssim& \hnorm{u_h - u} + \norm{p - \LtwoOp p} + \osc(f,h). 
  \end{align}
  Mit der Dreiecksungleichung erhält man dann die Abschätzung
  (\ref{eq:26}) und damit letztendlich die Behauptung von Satz
  \ref{thm:4}. 
\end{proof}

\end{document}
