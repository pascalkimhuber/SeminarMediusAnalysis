\message{ !name(script.tex)}\documentclass[a4paper]{scrartcl}

% --------------------------------------- Packages
\usepackage[utf8]{inputenc}
\usepackage{mathtools}
\usepackage{amssymb}
\usepackage{amsthm}
\usepackage{enumitem}
\usepackage{bbold}
\usepackage{draftwatermark}
\usepackage{xcolor}
\usepackage{esint}

% --------------------------------------- Definitions

\newcommand{\real}{\mathbb{R}}

\newcommand{\ubold}{\boldsymbol{u}}
\newcommand{\fbold}{\boldsymbol{f}}

\newcommand{\Hzero}{H_0^1}
\newcommand{\Ltwo}{L^2}
\newcommand{\crfem}{\operatorname{CR}_0^1}
\newcommand{\czero}{\operatorname{C}_0}
\newcommand{\J}[1]{J_{#1}}


\newcommand{\mesh}{\mathcal{T}_h}
\newcommand{\edges}{\mathcal{E}}
\newcommand{\nodes}{\mathcal{N}}

\newcommand{\dx}{\,dx}
\newcommand{\vnorm}[1]{\left\lVert#1\right\rVert_V}
\newcommand{\norm}[1]{\left\lVert#1\right\rVert}
\newcommand{\hnorm}[1]{\left\lVert#1\right\rVert_h}
\newcommand{\set}[1]{\left\{#1\right\}}

\DeclareMathOperator{\divOp}{div}
\DeclareMathOperator{\divh}{div_h}
\DeclareMathOperator{\nablah}{\nabla_{\textit{h}}}
\DeclareMathOperator{\diam}{diam}
\DeclareMathOperator{\midOp}{mid}
\DeclareMathOperator{\intOp}{I_{NC}}
\DeclareMathOperator{\LtwoOp}{\Pi_0}
\DeclareMathOperator{\conv}{conv}
\DeclareMathOperator{\osc}{osc}


% --------------------------------------- Environments
\theoremstyle{plain}
\newtheorem{theorem}{Theorem}
\newtheorem{lemma}{Lemma}
\theoremstyle{definition}
\newtheorem{definition}{Definition}
\newtheorem{modellproblem}{Modellproblem}
\theoremstyle{remark}
\newtheorem*{remark*}{Bemerkung}

% --------------------------------------- Some settings
\SetWatermarkScale{4}

% --------------------------------------- Document
\begin{document}

\message{ !name(script.tex) !offset(-3) }

\title{\textsc{CRFEM for the Stokes Problem}}
\author{Pascal Huber}
\date{June, 30, 2014}
\maketitle

\section{Motivation und Einleitung}
\label{sec:motiv-und-einl}

{\color{red}
  \begin{enumerate}
  \item Was wir bisher gelernt haben
  \item Was ich heute machen möchte
  \item Was wir dazu benutzen werden
  \end{enumerate}
}

\noindent Wir beginnen mit einem kleinen Überblick zu den Stokes Gleichungen. 

\section{Das Stokes Problem}
\label{sec:das-stokes-problem}

\paragraph{Schwache Formulierung des Stokes Problems}
\label{sec:schw-form-des}

Das Stokes Problem 
\begin{equation}
  \label{eq:1}
  -\Delta u + \nabla p = f \text{ in } \Omega \quad \text{ und }
  \quad \divOp u = 0 \text{ in } \Omega
\end{equation}
dient der Beschreibung von inkompressiblen
Flüssigkeiten mit hoher Viskosität (und ist damit eine linearisierte
Vereinfachung der Navier-Stokes-Gleichungen).\\ 

\noindent Im Folgenden betrachten wir als Grundlage für unsere spätere
Diskretisierung das Stokes Problem mit homogenen Randbedinungen auf
einem zwei dimensionalen Lipschitz-Gebiet: 

\begin{modellproblem}[Schwache Formulierung des Stokes Problems] \label{problem:1}
  Sei \(\Omega \subset \real^2\) ein polygonales beschränktes
  Lipschitz-Gebiet und \(f\in \Ltwo(\Omega; \real^2)\). Wir definieren
  die Räume 
  \[V \coloneqq \Hzero(\Omega; \real^2) \text{ und } Q \coloneqq
  \Ltwo_0(\Omega) \coloneqq \{q \in \Ltwo(\Omega): \int_\Omega q(x)
  \dx = 0\},\]
  sowie die Bilinearformen \(a \colon V\times V \rightarrow \real\)
  und \(b \colon V\times Q \rightarrow \real\) durch
  \begin{equation}
    \label{eq:2}
    a(u,v) \coloneqq \int_\Omega \nabla u : \nabla v \dx 
    \text{ und }
    b(v,q) \coloneqq \int_\Omega \divOp v \cdot q \dx. 
  \end{equation}
  Finde \((u,p) \in V\times Q\), sodass folgende Gleichungen erfüllt
  sind: 
  \begin{equation}
    \label{eq:3}
    \begin{cases}
      a(u,v) - b(v,p) = \langle f, v \rangle_{\Ltwo(\Omega;\real^2)} &
      \text{ für alle } v \in V \\
      b(u,q) = 0 & \text{ für alle } q \in Q.
    \end{cases}
  \end{equation}
\end{modellproblem}

{\color{blue}
  \noindent Gesucht werden also der Geschwindigkeitsvektor \(u \in V\) der
  Flüssigkeit und der dazugehörige Druck \(p \in Q\). Die Funktion \(f
  \in \Ltwo(\Omega, \real^2)\) beschreibt die gegebene äußere
  Kraftdichte, die auf die Flüssigkeit wirkt. \\
  Wir bemerken, dass es sich im Gegensatz zu den bisherigen Vorträgen um
  ein vektorwertiges Problem handelt. \\
}

\noindent Ein Problem von der Form (\ref{eq:3}) wird als Sattelpunktproblem
bezeichnet. Dabei wird die Gleichung \(b(u,q) = 0 \) als Nebenbedinung
zur eigentlichen Minimi\-sierungs\-aufgabe verstanden. \\ 

\paragraph{Existenz und Eindeutigkeit für Lösungen von Sattelpunktproblemen}
\label{sec:exist-und-eind}

\noindent Bevor wir zur Diskretisierung des Problems mit Hilfe von
Crouzeix-Raviart Elementen übergehen, widmen wir uns kurz der Existenz
und Eindeutigkeit einer Lösung \((u,p) \in V\times Q\) für das Modellproblem
\ref{problem:1}.\\
Da es sich hier um ein ``gemischtes Problem'' handelt, ist der
gewöhnliche Ansatz für den Existenzbeweis mit Hilfe des Lemmas von
Lax-Milgram nicht möglich. Stattdessen verwenden wir das
Splitting-Theorem von Brezzi, das notwendige und hinreichende
Bedingungen für die Existenz einer eindeutigen Lösung liefert. Dazu
betrachten wir eine etwas allgemeinere Fassung unseres Modellproblems.

\begin{modellproblem}[Sattelpunktproblem] \label{problem:2}
  Seien \((V,\vnorm{\cdot})\)  und \((Q,\norm{\cdot}_Q)\) zwei
  Hilbert-Räume und 
  \[a\colon V\times V \rightarrow \real 
  \quad \text{ und } \quad
  b\colon V\times Q \rightarrow \real\]
  zwei stetige Bilinearformen. Wir nehmen an, dass \(a(\cdot, \cdot)\) symmetrisch
  ist und definieren den abgeschlossenen Unterraum 
  \begin{equation}
    \label{eq:7}
    Z \coloneqq \left\{v \in V: b(v,q) = 0 \text{ for all } q\in Q\right\}.
  \end{equation}
  Weiter seien \(F \in V'\) und \(G \in Q'\)\\
  Finde \((u,p) \in V\times Q\), sodass folgende Gleichungen erfüllt
  sind: 
  \begin{equation}
    \label{eq:4}
    \begin{cases}
      a(u,v) - b(v,p) = F(v) & \text{ für alle } v \in V \\
      b(u,q) = G(q) & \text{ für alle } q \in Q.
    \end{cases}
  \end{equation}
\end{modellproblem}

\noindent Der folgende Satz garantiert Existenz und Eindeutigkeit der Lösung für
Modellproblem \ref{problem:2}. 

\begin{theorem}[Brezzis Splitting-Theorem] \label{thm:1}
  Die Abbildung \(L\colon V\times Q \rightarrow V' \times Q'\), die
  jedem Paar \((u,p) \in V\times Q\) gemäß dem Sattelpunktproblem
  (\ref{eq:4}) das Paar \((F,G) \in V'\times Q'\) zuordnet, ist genau
  dann ein Isomorphismus, wenn die beiden folgenden Bedingungen erfüllt
  sind: 
  \begin{enumerate}[label=(\roman*)]
  \item Die Bilinearform \(a(\cdot, \cdot)\) ist \(Z\)-elliptisch:
    \begin{equation}\label{eq:5}
      \exists \alpha > 0: \forall z \in Z: a(z,z) \geq \alpha \vnorm{z}^2.
    \end{equation}

  \item Die Bilinearform \(b(\cdot, \cdot)\) erfüllt die
    inf-sup-Bedingung: 
    \begin{equation}
      \label{eq:6}
      \exists \beta > 0: \inf_{q\in Q} \sup_{v\in V} \frac{b(v,q)}{\vnorm{v}\norm{q}_Q} \geq \beta.
    \end{equation}
  \end{enumerate}
\end{theorem}

\noindent Der Beweis dieses Satzes nutzt das \emph{closed range theorem} und
wird hier nicht angegeben. \\
Wir bemerken, dass wir statt der Elliptizität der Bilinearform \(b(\cdot,
\cdot)\) die inf-sup-Bedingung für Satz \ref{thm:1} fordern
müssen. Desweiteren weisen wir darauf hin, dass die Elliptizität von
\(a(\cdot, \cdot)\) nur auf dem Unterraum \(Z\) gefordert wird. \\


\paragraph{Existenz von Lösungen für das Stokes Problem}
\label{sec:exist-von-losung}

Zum Beweis der Existenz und Eindeutigkeit von Lösungen für das Stokes
Problem \ref{problem:1} genügt es also nach Theorem \ref{thm:1} die
\(Z\)-Elliptizität von \(a(\cdot, \cdot)\) und die 
inf-sup-Bedingung für \(b(\cdot, \cdot)\) nachweisen. (Die Stetigkeit
der beiden Bilinearformen folgt direkt aus der Cauchy-Schwarz
Ungleichung.)\\

\noindent Wie bereits in den letzten Vorträgen gesehen, folgt die
Elliptizität von \(a(\cdot, \cdot)\) auf ganz \(\Hzero(\Omega)\) aus der Friedrichs-Ungleichung. \\ 
Zum Beweis der inf-sup-Bedingung benötigen wir ein tiefgreifendes
analytisches Resultat, bekannt als Lady\u{z}enskaya Abschätzung: 

\begin{theorem}[Lady\u{z}enskaya Ungleichung] \label{thm:2}
  Sei \(\Omega \subset \real^d\) ein beschränktes
  Lipschitz-Gebiet. Dann ist die Abbildung 
  \[\divOp \colon Z^\perp \rightarrow \Ltwo_0(\Omega), \quad v \mapsto
  \divOp v\]
  ein Isomorphismus und es exisitiert ein Konstante \(c = c(\Omega) >
  0\), so dass gilt: 
  \[\forall q \in \Ltwo_0(\Omega)\colon \exists v_q \in Z^\perp \colon 
  \divOp v_q = q \text{ und } \norm{v_q}_{\Hzero(\Omega; \real^2)} \leq c \norm{q}_{\Ltwo(\Omega)}.\]
\end{theorem}

\noindent Aus Theorem \ref{thm:2} folgt unmittelbar die
inf-sup-Bedingung für \(b(\cdot, \cdot)\): Für alle \(q\in Q\) gilt: 
\begin{equation}
  \label{eq:8}
  \sup_{v\in V} \frac{b(v,q)}{\vnorm{v}} \geq \frac{b(v_q,q)}{\vnorm{v_q}} = \frac{\norm{q}^2_Q}{\vnorm{v_q}} \geq \frac{1}{c} \norm{q}_Q,  
\end{equation}
wobei wir im vorletzten Schritt die explizite Definition von
\(b(\cdot, \cdot)\)  ausgenutzt haben. \\

{\color{red}
  \noindent Fragen: 
  \begin{enumerate}
  \item Normen von \(V\) und \(Q\) angeben?
  \item Definition von \(A:B\) angeben?
  \item Was genau bedeuted die inf-sup-Bedingung?
  \item Soll ich die Ladyzenskaya Ungleichung rauslassen? 
  \end{enumerate}
}


\section{Stabilität der CRFEM für das Stokes Problem}
\label{sec:stabilitat-der-crfem}

\subsection{Motivation}
\label{sec:motivation}

{\color{red}
  \begin{enumerate}
  \item Was ist die generelle Schwierigkeit bei der Diskretisierung
    des Stokes-Problems?
  \item Warum scheint das CR-Element gut geeignet zu sein? (Das könnte
    ich auch nach der Definition des diskreten Problems machen...)
  \end{enumerate}
}


\subsection{Beschreibung der Diskretisierung}
\label{sec:beschr-der-diskr}

Nachdem wir nun die Existenz und Eindeutigkeit einer Lösung für das
Stokes Problem sichergestellt haben, wollen wir im Folgenden die
Diskretisierung von Modellproblem \ref{problem:1} mit Hilfe des
Crouzeix-Raviart Elements beschreiben. \\

\noindent Sei dazu \(\mesh\) ein quasi-uniforme Triangulierung von
\(\Omega\). Zur Beschreibung der Triangulierung benutzen wir folgende
Bezeichnungen: 
\begin{itemize}
\item \(h \coloneqq \max\{h_T\colon T \in \mesh\}\) mit \(h_T \coloneqq \diam(T)\) für \(T\in \mesh\)
\item \(\edges\) bezeichnet die Menge aller Kanten von \(\mesh\)
\item \(\midOp(E)\) bezeichnet den Kantenmittelpunkt von \(E \in
  \edges\), \(\midOp(\edges)\) die Menge aller Kantenmittelpunkte in \(\mesh\)
\end{itemize}

\noindent Zur Definitionen der Ansatzräume für die Diskretisierung
benötigen wir noch folgende endlich-dimensionalen Funktionenräume: \\
Es bezeichne 
\[P_k(\mesh) \coloneqq \set{v_h \in \Ltwo(\Omega)\colon \forall T \in
  \mesh\colon v_h|_T \in P_k(T)}\]
den Raum aller stückweiser Polynome vom Grad höchstens \(k\) bezüglich
der Triangulierung \(\mesh\). Wir definieren den \emph{Crouzeix-Raviart
FE-Raum} durch 
\begin{equation}
  \label{eq:9}
  \crfem(\mesh) \coloneqq \set{v_h \in P_1(\mesh)\colon v_h \text{ stetig in } \midOp(\edges), v_h(\midOp(E)) = 0 \text{ für } E \in \partial\Omega\cap\edges}.
\end{equation}

\noindent Mit Hilfe dieser Definitionen sind wir nun in der Lage die
nicht-konforme Crouzeix-Raviart Diskretisierung des Modellproblems
\ref{problem:1} anzugeben. 

\begin{modellproblem}[CR-NCFEM für das Stokes
  Modellproblem] \label{problem:3}
  Wir definieren die Ansatzräume \(V_h \coloneqq \crfem(\mesh) \times
  \crfem(\mesh)\) und \(Q_h \coloneqq \Ltwo_0(\Omega) \cap
  P_0(\mesh)\) zusammen mit den Normen 
  \begin{equation}
    \label{eq:10}
    \norm{q_h} \coloneqq \norm{q_h}_{\Ltwo(\Omega)} \text{ auf } Q_h
    \quad \text{ und } \quad
    \hnorm{v_h} \coloneqq \norm{\nablah v_h} \text{ auf } V_h. 
  \end{equation}
  Weiter seien die diskreten Bilinearformen \(a_h\colon V_h \times V_h
  \rightarrow \real\) und \(b_h\colon V_h \times Q_h \rightarrow
  \real\) definiert durch
  \begin{equation}
    \label{eq:11}
    a_h(u_h, v_h) \coloneqq \int_\Omega \nablah u : \nablah v \dx
    \quad \text{ und } \quad b_h(v_h, q_h) \coloneqq \int_\Omega q_h\divh v_h \dx. 
  \end{equation}
  Für gegebenes \(f \in \Ltwo(\Omega; \real^2)\) finde \((u_h, p_h)
  \in V_h \times Q_h\), sodass folgende Gleichungen gelten: 
  \begin{equation}
    \label{eq:12}
    \begin{cases}
      a_h(u_h, v_h) - b_h(v_h, p_h) = \langle f, v_h \rangle_{\Ltwo(\Omega; \real^2)} & \text{ für alle } v_h \in V_h \\
      b_h(u_h, q_h) = 0 & \text{ für alle } q_h \in Q_h. 
    \end{cases}
  \end{equation}
  \end{modellproblem}

{\color{red}
\noindent Fehlende Bemerkungen: 
\begin{itemize}
\item Definition von \(\nablah\) und \(\divh\).
\item \(\hnorm{\cdot}\) heißt Energienorm und ist tatsächlich eine
  Norm auf \(V_h\) (siehe Benjamins Vortrag).
\item Auf \(\Hzero(\Omega; \real^2)\) stimmt \(\hnorm{\cdot}\) mit
  \(\norm{\cdot}_{\Hzero}\) überein.
\item Die Methode ist nicht-konform, da offensichtlich \(V_h
  \not\subseteq V\).
\item Motivation: Warum eigentlich CR-NCFEM? \(\Rightarrow\) einfache Struktur,
  inf-sup-Stabilität, Approximationseigenschaften
\end{itemize}
}



\subsection{Stabilität der CR-NCFEM}
\label{sec:stabilitat-der-cr}

Bevor wir mit der Fehlerabschätzung für die
Crouzeix-Raviart-Diskretisierung beginnen können, muss noch gezeigt
werden, dass im Modellproblem \ref{problem:3} auch tatsächlich eine
eindeutige Lösung \((u_h,p_h)\) exisitiert. \\

\noindent Dazu verwenden wir wie auch schon für das kontinuierliche Problem die
Aussage von Theorem \ref{thm:1}. Man kann sich leicht davon
überzeugen, dass durch die Wahl der Normen \(\hnorm{\cdot}\) und
\(\norm{\cdot}\) die Stetigkeit beider Bilinearformen und auch die
Elliptizität von \(a(\cdot, \cdot)\) auf dem Unterraum \(Z_h \coloneqq
\set{v_h \in V_h\colon \forall q_h \in Q_h\colon b_h(v_h, q_h) = 0}\)
(mit einer von \(h\) unabhängingen Konstante) gewährleistet ist. \\

\noindent Also besteht das eigentliche Kriterium für die Existenz
einer eindeutigen Lösung wie schon im ursprünglichen Problem in der
inf-sup-Bedingung. Auch darüber hinaus spielt die inf-sup-Bedingung
in der Praxis für die Stabilität bei der numerischen Berechnung eine
wichtige Rolle. Dies motiviert die folgende Defintion: 

\begin{definition}[Stabilität von FEM für das Stokes Problem] \label{def:1}
  Eine FE-Diskretisierung \(V_h\times Q_h\) für das Modellproblem
  \ref{problem:1} heißt \emph{inf-sup-stabil} (oder einfach nur
  \emph{stabil}), falls die diskrete inf-sup-Bedingung erfüllt ist,
  d.h. es existiert \(\beta > 0\) (unabhänging von der Gitterweite
  \(h\)), sodass 
  \begin{equation}
    \label{eq:14}
    \inf_{q_h \in Q_h} \sup_{v_h \in V_h} \frac{b_h(v_h, q_h)}{\hnorm{v_h} \norm{q_h}} \geq \beta. 
  \end{equation}
  \end{definition}

\noindent Im Folgenden wollen wir nun zeigen, dass die Stabilität für die
Crouzeix-Raviart-FEM gegeben ist. Dazu verwenden wir das sogenannte
\emph{Fortin-Kriterium}, indem wir einen geeigneten nicht-konformen
Interpolationsoperator definieren. {\color{red}(Diesen werden wir auch noch im
Laufe der Medius-Analyse benötigen...)}

\begin{definition}[Nicht-konformer Interpolationsoperator und
  \(\Ltwo\)-Projektion] \label{def:2}
  \begin{enumerate}[label=\emph{(\roman*)}]
  \item Der nicht-konformen Interpolationsoperator \(\intOp\colon
    \Hzero(\Omega) \rightarrow \crfem(\mesh)\) ist definiert durch
    \begin{equation}
      \label{eq:15}
      (\intOp v)(\midOp(E)) \coloneqq \fint_E v \,ds \qquad \text{für } E \in \edges. 
    \end{equation}
  \item Wir definieren \(\LtwoOp \colon \Ltwo(\Omega) \rightarrow
    P_0(\mesh)\) durch 
    \begin{equation}
      \label{eq:13}
      (\LtwoOp f)|_T \coloneqq \fint_T f \dx \qquad \text{für } T\in \mesh. 
    \end{equation}
  \end{enumerate}
\end{definition}

Die Besonderheit von \(\intOp\) und \(\LtwoOp\) liegt vor allem
darin, dass sie jeweils die Bestapproximation ihres Arguments im
Bildraum darstellen: 

\begin{lemma}[Eigenschaften von \(\intOp\) und \(\LtwoOp\)] \label{lem:1}
  \begin{enumerate}[label=(\roman*)]
  \item Es gelten die folgenden Mittelwerteigenschaften für alle 
    \(T\in\mesh\) und \(v\in V\), \(f\in\Ltwo(\Omega)\): 
    \begin{equation}
      \label{eq:16}
      \fint_T \nabla v \dx = \fint_T \nabla(\intOp v) \dx \quad \text{und} \quad
      \fint_T f \dx = \fint_T \LtwoOp f \dx 
    \end{equation}
  \item Seien \(v\in V\) und \(f\in \Ltwo(\Omega)\) gegeben. Dann gilt
    für alle \(w\in P_0(\mesh)^2\) und alle \(g\in P_0(\mesh)\): 
    \begin{equation}
      \label{eq:17}
      \norm{\nablah(v-\intOp v)} \leq \norm{\nabla v - w} \quad
      \text{und} \quad \norm{f-\LtwoOp f} \leq \norm{f-g}. 
    \end{equation}
  \end{enumerate}
\end{lemma}

\begin{proof}[Beweis.]
  \begin{enumerate}[label=\emph{(\roman*)}]
  \item Die Gleichheit der Mittelwerte in (\ref{eq:16}) folgt für
    \(\LtwoOp\) direkt aus der Definition. Für \(\intOp\) erhält man die
    Aussage mit Hilfe von partieller Integration (bemerke, dass wir hier
    den Spursatz verwenden).
  \item Wir zeigen die Approximationseigenschaft für \(\intOp\): Sei
    \(v \in V\), \(w\in P_0(\mesh)^2\). Dann gilt: 
    \begin{align}
      \norm{\nabla v - w}^2 =& \int_\Omega \left(\nablah(v - \intOp v)\right)^2 + \left(\nablah \intOp v - w\right)^2 \dx \nonumber \\ 
      & + \int_\Omega \left(\nablah(v - \intOp v)\right) \cdot \left(\nablah \intOp v - w\right) \dx. \nonumber
    \end{align}
    Da aber \(\left(\nablah \intOp v - w\right)\) stückweise konstant
    ist, folgt mit der Mittelwerteigenschaft von (\ref{eq:16}) die
    Behauptung. \\
    Die Ungleichung für \(\LtwoOp\) geht analog. 
\end{enumerate}
\end{proof}


{\color{red}
  \begin{itemize}
  \item Erwähne dass für vektorwertige Funktionen die Anwendung
    komponentenweise geschieht.
  \item Was ist mit dem Beweis von Lemma \ref{lem:1}? 
  \end{itemize}
}

\noindent Wir können nun den nicht-konformen Interpolationsoperator \(\intOp\)
verwenden, um die Stabilität der Crouzeix-Raviart FE-Räume zu zeigen. 

\begin{theorem}[Stabilität von CR-NCFEM]\label{thm:3}
  Die Crouzeix-Raviart Diskretisierung des Stokes Problems \(V_h \times Q_h =
  \crfem(\mesh)^2 \times (\Ltwo_0(\Omega)\cap P_0(\mesh))\) aus
  Modellproblem \ref{problem:3}  ist stabil. 
\end{theorem}

\begin{proof}[Beweis.]
  Für den Beweis zeigen, wir dass der nicht-konforme
  Interpolationsoperator ein \emph{Fortin-Operator} ist, d.h. wir
  zeigen: 
 \begin{align}     
   &(i) \quad \exists c > 0 \text{ (unabhängig von } h\text{)} \colon
   \hnorm{\intOp v} \leq c \vnorm{v} \text{ für alle } v \in V \nonumber\\ 
   &(ii) \quad \forall v \in V, \forall q_h \in Q_h\colon b_h(\intOp v, q_h) = b(v, q_h) \nonumber
   \label{align:2} 
  \end{align}
  Daraus folgt dann die inf-sup-Bedingung. \\
  Da \(\nablah (\intOp v)\) stückweise konstant ist, folgt mit
  (\ref{eq:16})
  \begin{equation}
    \label{eq:18}
    \int_\Omega \nablah (\intOp v) : \nablah (\intOp v) \dx = 
    \int_\Omega \nabla v : \nablah (\intOp v) \dx \leq \norm{\nabla v}\hnorm{\intOp v}
  \end{equation}
  und damit \emph{(i)}. Analog folgt \emph{(ii)} auch aus
  der Mittelwerteigenschaft, da \(q_h \in Q_h\) stückweise konstant
  ist. \\ 
  Sei nun \(q_h \in Q_h\) beliebig. Dann erhält man aus der
  inf-sup-Bedingung von \(b(\cdot, \cdot)\) und \emph{(i), (ii)}: 
  \begin{align}
    \beta \norm{q_h} 
    &\leq \sup_{v\in V} \frac{b(v,q_h)}{\vnorm{v}}
    \overset{\emph(ii)}{=} \sup_{v\in V} \frac{b_h(\intOp v,q_h)}{\vnorm{v}}
    \overset{\emph(i)}{\leq} \sup_{v\in V} \frac{b_h(\intOp v,q_h)}{\hnorm{\intOp v}} \nonumber 
    \leq \sup_{v_h\in V_h} \frac{b_h(v_h,q_h)}{\hnorm{v_h}}. \nonumber
  \end{align}
\end{proof}

\begin{remark*}
  Wir haben gesehen, dass die Ungleichung in \emph{(i)} mit der
  bestmöglichen Konstante \(c=1\) für den nicht-konformen
  Interpolationsoperator \(\intOp\) gilt. Dies ist auch Hinweis
  darauf, dass das CR-Element sehr gut für das Stokes-Problem geeignet
  ist. 
\end{remark*}



\section{Medius Analyse der CRFEM für das Stokes Problem}
\label{sec:medius-analyse-der}

\subsection{Motivation}
\label{sec:motivation-1}

In den vorangegangen Abschnitten haben wir gesehen, dass die
Crouzeix-Raviart Diskretisierung eine relativ einfache FEM für das
Stokes Problem darstellt, die zudem sehr gute Stabilitätseigenschaften
aufweist. \\
Zum Abschluss des Vortrags wollen wir für die Fehleranalyse ein
Resultat zur Bestapproximation der Crouzeix-Raviart Diskretisierung
angeben. \\

\noindent Anstatt einer Fix-Strang Zerlegung in einen
Approximations- und einen Konsistenzfehler, wollen wir die
Fehlerabschätzung mit Hilfe der Medius Analyse herleiten. Dies hat den
Vorteil, dass für die Lösung \((u,p)\) des Stokes Problems keine
weiteren Regularitätsannahmen gemacht werden müssen. \\

\noindent Bevor wir zur Aussage über die Bestapproximation und deren
Beweis kommen, führen wir noch ein paar notationelle Vereinfachungen
ein. 

\subsection{Notation und konforme Begleitabbildungen}
\label{sec:notat-und-konf}

\paragraph{Notation}
\label{sec:notation}

Wir verwenden folgende Notation: 
\begin{itemize}
\item Wir schreiben \(a \lesssim b\), falls es eine von der
  Gitterweite \(h\) unabhänginge Konstante \(C > 0\) gibt, sodass \(a
  \leq C b\).
\item Wie schon in Modellproblem \ref{problem:3} verwenden wir die
  Bezeichnungen \(V_h \coloneqq \crfem(\mesh)\times\crfem(\mesh)\) und \(Q_h
  \coloneqq P_0(\mesh)\cap\Ltwo_0(\Omega)\) und definieren auf diesen
  Räumen die Normen 
  \[\norm{f} \coloneqq \norm{f}_{\Ltwo(\Omega;\real^m)} \quad \text{
    und } \quad
  \hnorm{v_h} \coloneqq \norm{\nablah v_h} \quad \text{für } f\in\Ltwo(\Omega; \real^m), v_h \in V_h.\]
\item Mit \(\LtwoOp\) und \(\intOp\) bezeichnen wir die
  \(\Ltwo\)-Projektion und den nicht-konformen Interpolationsoperator
  aus Definition \ref{def:2}. 
\end{itemize}

\paragraph{Konforme Begleitabbildungen}
\label{sec:konf-begl}

Wie schon im letzten Vortrag zur Medius Analyse der CRFEM für das
Poisson Problem verwenden wir für den Beweis der
Approximationabschätzung eine Abbildung der Form 
\[ J_k \colon \crfem(\mesh) \rightarrow
P_k(\mesh)\cap\czero(\Omega).\] 
Diese stellt einen Zusammenhang zwischen dem nicht-konformen
Crouzeix-Raviart Raum und dem konformen FE-Raum aller stückweiser
Polynome vom Grad höchstens \(k\) her. Eine
solche Abbildung \(J_k\) werden wir im Folgenden \emph{(konforme)
  Begleitabbildungen} nennen. \\ 
Es stellt sich heraus, dass wir für den Beweis der Bestapproximation
eine Begleitabbildungen vom Grad \(k=3\) benötigen, die wir nun in
drei Schritten als Mittelungsoperatoren definieren wollen: 

\begin{definition}[Konforme Begleitabbildung \(J_3\)]\label{def:3}
  Sei \(\mesh\) eine quasi-uniforme Triangulierung von \(\Omega\). Es
  bezeichne 
  \begin{itemize}
  \item \(\varphi_z\) die konforme nodale Basisfunktion bezüglich
    eines Knotens \(z\in\mathcal{N}\),
  \item \(b_E \coloneqq 6\varphi_a\varphi_b\) die Blasenfunktion
    bezüglich der Kante \(E \coloneqq 
    \conv\{a,b\} \in \edges\) und
  \item \(b_T \coloneqq 60\varphi_a\varphi_b\varphi_c\) die
    Blasenfunktion für das Element \(T \coloneqq \conv\{a,b,c\} \in
    \mesh\).
  \end{itemize}
  Dann definieren wir die konformen Begleitabbildungen \(J_k \colon
  \crfem(\mesh) \rightarrow P_k(\mesh)\cap\czero(\Omega)\) für
  \(k=1,2,3\) wie folgt für \(v_h \in \crfem(\mesh)\): 
  \begin{enumerate}[label=\textit{(\roman*)}]
  \item \(J_1 v_h \coloneqq \sum_{z \in
      \nodes(\Omega)}\left(\left\lvert \mathcal{T}(z)\right\rvert^{-1}
      \sum_{T\in \mathcal{T}(z)} v_h\vert_T(z) \varphi_z \right)\)
  \item \(J_2 v_h \coloneqq J_1 v_h + \sum_{E\in\edges(\Omega)} 
    \left(\fint_E (v_h - J_1 v_h) \,ds\right)b_E\)
  \item \(J_3 v_h \coloneqq J_2 v_h + \sum_{T\in\mesh}\left(\fint_T(v_h
      - J_2 v_h)\dx \right)b_T\)
  \end{enumerate}
\end{definition}

{\color{red}
  \begin{itemize}
  \item Definiere, was \(\mathcal{N}(\Omega)\) und \(\edges(\Omega)\) ist!!
  \end{itemize}
}

\begin{remark*}
  Durch komponentenweise Anwendung definieren \(J_k\) auf natürliche
  Weise Abbildungen auf \(V_h\). \\
  Für den Beweis der Bestapproximation werden wir nur \(J_3\)
  verwenden. 
\end{remark*}

\noindent Die Motivation für die Definition der Begleitabbildungen als passende
Mittelungen ist im Wesentlichen durch das folgende Lemma gegeben, das
wir auch im späteren Hauptbeweis des öfteren benutzen werden. 

\begin{lemma}[Eigenschaften von \(J_3\)]\label{lem:2}
  Die konforme Begleitabbildung \(\J3\colon \crfem(\mesh) \rightarrow
  P_3(\mesh)\cap\czero(\Omega)\) hat folgende Eigenschaften: 
  \begin{enumerate}[label=(\roman*)]
  \item \(\int_T \nablah(v_h - \J3 v_h)\dx = 0 \quad\) für \(v_h \in
    \crfem(\mesh), T\in \mesh\),
  \item \(\int_T v_h - \J3 v_h \dx = 0 \quad\) für \(v_h \in
    \crfem(\mesh), T\in \mesh\),
  \item \(\norm{\nablah (v_h - \J3 v_h)} \lesssim \norm{h^{-1}(v_h -
      \J3 v_h)} \lesssim \norm{\nablah v_h} \quad\) für alle \(v_h \in
    \crfem(\mesh)\).
  \end{enumerate}
\end{lemma}

{\color{red}
  \begin{itemize}
  \item Was ist mit dem Beweis hierzu?
  \item Soll ich noch einmal eine kleine Übersicht zu allen wichtigen
    Relationen machen? 
  \end{itemize}
}


\subsection{Bestapproximation der CRFEM für das Stokes Problem}
\label{sec:best-der-crfem}

Mit diesen Hilfsmitteln sind wir nun in der Lage das folgende Resultat
zur Bestapproximation der CRFEM für das Stokes Problem zu beweisen. 
Wir weisen noch einmal darauf hin, dass für das Resultat keine
weiteren Regularitätsannahmen für die Lösung des Problems gemacht
werden muss. Dies ist ein weiteres Argument, dass für die
Diskretisierung des Stokes Problems durch die CRFEM spricht. 

\begin{theorem}[Bestapproximation der CRFEM]\label{thm:4}
  Sei \((u,p) \in V\times Q\) die Lösung von Problem \ref{problem:1}
  und \((u_h,p_h) \in V_h\times Q_h\) die Lösung von Problem
  \ref{problem:3}. Dann gilt für alle \(v_h \in V_h\) und \(q_h \in
  Q_h\)
  \begin{equation}
    \label{eq:19}
    \hnorm{u - u_h} + \norm{p - p_h} \lesssim \hnorm{u - v_h} + \norm{p - q_h} + \osc(f,h), 
  \end{equation}
  wobei die Oszillation durch  \(\osc(f, h) \coloneqq \norm{h(f -
    \LtwoOp f)}\) definiert ist. 
\end{theorem}

Den Beweis führen wir in zwei Schritten:
\begin{enumerate}
\item Für die Abschätzung des Geschwindigkeitsterms \(\hnorm{u -
    u_h}\) verwenden wir die Eigenschaften der Begleitabbildungen
  \(\J3\) und nutzen aus, dass die nicht-konforme Interpolation
  \(\intOp\) die Bestapproximation an \(u\) im Raum \(V_h\) darstellt.
\item Die Abschätzung des Druckterms \(\norm{p - p_h}\) leitet sich
  Wesentlichen aus der Stabilität des Crouzeix-Raviart Elements her. 
\end{enumerate}

\begin{proof}[Beweis.]
  \textsf{\textbf{\\Teil 1:}}\\
  Wir wollen den Term den Fehler \(\hnorm{u - u_h}\) gegen \(\hnorm{u
    - \intOp u}\) abschätzen. Aus der Definition von \(a_h(\cdot,
  \cdot)\) folgt
  \begin{equation}
    \label{eq:20}
    \hnorm{u - u_h}^2 = a_h(u - u_h, u - \intOp u) + a_h(u - u_h, \intOp u - u_h). 
  \end{equation}
  Wir setzen \(w_h \coloneqq \intOp u - u_h \in V_h\). Aus \(\divOp u
  = 0\) und \(\divh u_h = 0\) erhält man mit der Mittelwerteigenschaft
  von \(\intOp\), dass auch \(div_h w_h = 0\) punktweise gilt. 
\end{proof}



% In diesem Abschnitt soll zunächst eine Beschreibung des Stokes Problems
% gegeben werden. \\

% \noindent Dazu beginnen wir mit der starken Formulierung des Problems
% und geben anschließend eine gemischte Variationsformulierung an, die
% dann später auch als Grundlage für die Diskretisierung mit Hilfe des
% Crouzeix-Raviart Elements dienen soll.  


% \subsection{Starke Formulierung des Stokes Problems}
% \label{sec:starke-form-der}

% Die Stokesschen Gleichungen beschreiben die Bewegungen einer
% inkompressiblen Flüssigkeit in einem Körper. \\

% \noindent Als Modellproblem betrachten wir das Stokes Problem mit
% homogenen Dirichletbedingungen auf einem polygonalem
% Lipschitz-Gebiet. 

% \begin{modellproblem} \label{problem:1}
%   Sei \(\Omega \subset \real^2\) ein polygonales Lipschitz-Gebiet und
%   \(\fbold \in C^0(\Omega; \real^2)\). \\
%   Finde \(\ubold \in C^2(\Omega; \real^2) \cap C^0(\bar{\Omega}; \real^2)\)
%   und \(p \in C^1(\Omega)\) so dass gilt: 
%   \begin{equation} 
%     \label{eq:1}
%     \begin{cases}
%       - \Delta \ubold + \nabla p = \fbold  & \text{ in } \Omega, \\
%       \divOp \ubold = 0 & \text{ in } \Omega, \\
%       \ubold = 0 & \text{ auf } \partial \Omega. 
%     \end{cases}
%   \end{equation}
%   Da durch diese Angaben der Druck \(p\) nur bis auf eine Konstante
%   festgelegt wird, fordern wir zusätzlich
%   \[\int_\Omega p(x) \dx = 0.\]\\

%   \noindent Dabei beschreibt \(\ubold\) den Geschwindigkeitsvektor der
%   Flüssigkeit und \(p\) den Druck. Die Funktion \(\fbold\) ist die gegebene
%   äußere Kraftdichte. 
% \end{modellproblem}


% \subsection{Gemischte Variationsformulierung des Stokes Problems}
% \label{sec:gemischte-vari-des}

% Durch Multiplikation mit geeigneten Testfunktionen und Integration
% über das Gebiet \(\Omega\) erhält man die schwache Formulierung des
% Modellproblems \ref{problem:1}.  

% \begin{modellproblem} \label{problem:2}
%   Sei \(\Omega \subset \real^2\) ein polygonales Lipschitz-Gebiet und
%   \(f\in \Ltwo(\Omega; \real^2)\). Wir
%   definieren die Räume
%   \[V \coloneqq \Hzero(\Omega; \real^2) \text{ und } Q \coloneqq
%   \Ltwo_0(\Omega) \coloneqq \{q \in \Ltwo(\Omega): \int_\Omega q(x)
%   \dx = 0\},\]
%   sowie die Bilinearformen \(a: V\times V \rightarrow \real\)
%   und \(b: V\times Q \rightarrow \real\) durch
%   \begin{equation}
%     \label{eq:2}
%     a(u,v) \coloneqq \int_\Omega \nabla u \colon \nabla v \dx 
%     \text{ und }
%     b(v,q) \coloneqq \int_\Omega \divOp v \cdot q \dx. 
%   \end{equation}
%   Finde \((u,p) \in V\times Q\), sodass folgende Gleichungen erfüllt
%   sind: 
%   \begin{equation}
%     \label{eq:3}
%     \begin{cases}
%       a(u,v) - b(v,p) = \langle f, v \rangle_{\Ltwo(\Omega;\real^2)} &
%       \text{ für alle } v \in V \\
%       b(u,q) = 0 & \text{ für alle } q \in Q.
%     \end{cases}
%   \end{equation}
% \end{modellproblem}

% \noindent Ein Problem von der Form (\ref{eq:3}) bezeichnet man auch als
% Sattelpunktproblem. \textcolor{red}{Warum??}

% \noindent Die Existenz und Eindeutigkeit der Lösung \((u,p)\) lässt
% sich mit Hilfe des Splitting-Theorems von Brezzi zeigen. 


\end{document}

\message{ !name(script.tex) !offset(-784) }
