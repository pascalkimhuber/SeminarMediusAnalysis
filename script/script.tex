\documentclass[a4paper]{scrartcl}

% --------------------------------------- Packages
\usepackage[utf8]{inputenc}
\usepackage{mathtools}
\usepackage{amssymb}
\usepackage{amsthm}
\usepackage{enumitem}
\usepackage{bbold}
\usepackage{draftwatermark}
\usepackage{xcolor}

% --------------------------------------- Definitions

\newcommand{\real}{\mathbb{R}}

\newcommand{\ubold}{\boldsymbol{u}}
\newcommand{\fbold}{\boldsymbol{f}}

\newcommand{\Hzero}{H_0^1}
\newcommand{\Ltwo}{L^2}

\newcommand{\dx}{\,dx}
\newcommand{\vnorm}[1]{\left\lVert#1\right\rVert_V}
\newcommand{\norm}[1]{\left\lVert#1\right\rVert}
\newcommand{\hnorm}[1]{\left\lVert#1\right\rVert_h}

\DeclareMathOperator{\divOp}{div}
\DeclareMathOperator{\divh}{div_h}
\DeclareMathOperator{\nablah}{\nabla_h}


% --------------------------------------- Environments
\theoremstyle{plain}
\newtheorem{theorem}{Theorem}
\newtheorem{lemma}{Lemma}
\theoremstyle{definition}
\newtheorem{definition}{Definition}
\newtheorem{modellproblem}{Modellproblem}
\theoremstyle{remark}
\newtheorem*{remark*}{Remark}

% --------------------------------------- Some settings
\SetWatermarkScale{4}

% --------------------------------------- Document
\begin{document}
\title{\textsc{CRFEM for the Stokes Problem}}
\author{Pascal Huber}
\date{June, 30, 2014}
\maketitle

\section{Motivation und Einleitung}
\label{sec:motiv-und-einl}

{\color{red}
  \begin{enumerate}
  \item Was wir bisher gelernt haben
  \item Was ich heute machen möchte
  \item Was wir dazu benutzen werden
  \end{enumerate}
}

\noindent Wir beginnen mit einem kleinen Überblick zu den Stokes Gleichungen. 

\section{Das Stokes Problem}
\label{sec:das-stokes-problem}

\paragraph{Schwache Formulierung des Stokes Problems}
\label{sec:schw-form-des}

Das Stokes Problem 
\begin{equation}
  \label{eq:1}
  -\Delta u + \nabla p = f \text{ in } \Omega \quad \text{ und }
  \quad \divOp u = 0 \text{ in } \Omega
\end{equation}
dient der Beschreibung von inkompressiblen
Flüssigkeiten mit hoher Viskosität (und ist damit eine linearisierte
Vereinfachung der Navier-Stokes-Gleichungen).\\ 

\noindent Im Folgenden betrachten wir als Grundlage für unsere spätere
Diskretisierung das Stokes Problem mit homogenen Randbedinungen auf
einem zwei dimensionalen Lipschitz-Gebiet: 

\begin{modellproblem}[Schwache Formulierung des Stokes Problems] \label{problem:1}
  Sei \(\Omega \subset \real^2\) ein polygonales Lipschitz-Gebiet und
  \(f\in \Ltwo(\Omega; \real^2)\). Wir
  definieren die Räume
  \[V \coloneqq \Hzero(\Omega; \real^2) \text{ und } Q \coloneqq
  \Ltwo_0(\Omega) \coloneqq \{q \in \Ltwo(\Omega): \int_\Omega q(x)
  \dx = 0\},\]
  sowie die Bilinearformen \(a \colon V\times V \rightarrow \real\)
  und \(b \colon V\times Q \rightarrow \real\) durch
  \begin{equation}
    \label{eq:2}
    a(u,v) \coloneqq \int_\Omega \nabla u : \nabla v \dx 
    \text{ und }
    b(v,q) \coloneqq \int_\Omega \divOp v \cdot q \dx. 
  \end{equation}
  Finde \((u,p) \in V\times Q\), sodass folgende Gleichungen erfüllt
  sind: 
  \begin{equation}
    \label{eq:3}
    \begin{cases}
      a(u,v) - b(v,p) = \langle f, v \rangle_{\Ltwo(\Omega;\real^2)} &
      \text{ für alle } v \in V \\
      b(u,q) = 0 & \text{ für alle } q \in Q.
    \end{cases}
  \end{equation}
\end{modellproblem}

{\color{blue}
  \noindent Gesucht werden also der Geschwindigkeitsvektor \(u \in V\) der
  Flüssigkeit und der dazugehörige Druck \(p \in Q\). Die Funktion \(f
  \in \Ltwo(\Omega, \real^2)\) beschreibt die gegebene äußere
  Kraftdichte, die auf die Flüssigkeit wirkt. \\
  Wir bemerken, dass es sich im Gegensatz zu den bisherigen Vorträgen um
  ein vektorwertiges Problem handelt. \\
}

\noindent Ein Problem von der Form (\ref{eq:3}) wird als Sattelpunktproblem
bezeichnet. Dabei wird die Gleichung \(b(u,q) = 0 \) als Nebenbedinung
zur eigentlichen Minimi\-sierungs\-aufgabe verstanden. \\ 

\paragraph{Existenz und Eindeutigkeit für Lösungen von Sattelpunktproblemen}
\label{sec:exist-und-eind}

\noindent Bevor wir zur Diskretisierung des Problems mit Hilfe von
Crouzeix-Raviart Elementen übergehen, widmen wir uns kurz der Existenz
und Eindeutigkeit einer Lösung \((u,p) \in V\times Q\) für das Modellproblem
\ref{problem:1}.\\
Da es sich hier um ein ``gemischtes Problem'' handelt, ist der
gewöhnliche Ansatz für den Existenzbeweis mit Hilfe des Lemmas von
Lax-Milgram nicht möglich. Stattdessen verwenden wir das
Splitting-Theorem von Brezzi, das notwendige und hinreichende
Bedingungen für die Existenz einer eindeutigen Lösung liefert. Dazu
betrachten wir eine etwas allgemeinere Fassung unseres Modellproblems.

\begin{modellproblem}[Sattelpunktproblem] \label{problem:2}
  Seien \((V,\vnorm{\cdot})\)  und \((Q,\norm{\cdot}_Q)\) zwei
  Hilbert-Räume und 
  \[a\colon V\times V \rightarrow \real 
  \quad \text{ und } \quad
  b\colon V\times Q \rightarrow \real\]
  zwei stetige Bilinearformen. Wir nehmen an, dass \(a(\cdot, \cdot)\) symmetrisch
  ist und definieren den abgeschlossenen Unterraum 
  \begin{equation}
    \label{eq:7}
    Z \coloneqq \left\{v \in V: b(v,q) = 0 \text{ for all } q\in Q\right\}.
  \end{equation}
  Weiter seien \(F \in V'\) und \(G \in Q'\)\\
  Finde \((u,p) \in V\times Q\), sodass folgende Gleichungen erfüllt
  sind: 
  \begin{equation}
    \label{eq:4}
    \begin{cases}
      a(u,v) - b(v,p) = F(v) & \text{ für alle } v \in V \\
      b(u,q) = G(q) & \text{ für alle } q \in Q.
    \end{cases}
  \end{equation}
\end{modellproblem}

\noindent Der folgende Satz garantiert Existenz und Eindeutigkeit der Lösung für
Modellproblem \ref{problem:2}. 

\begin{theorem}[Brezzis Splitting-Theorem] \label{thm:1}
  Die Abbildung \(L\colon V\times Q \rightarrow V' \times Q'\), die
  jedem Paar \((u,p) \in V\times Q\) gemäß dem Sattelpunktproblem
  (\ref{eq:4}) das Paar \((F,G) \in V'\times Q'\) zuordnet, ist genau
  dann ein Isomorphismus, wenn die beiden folgenden Bedingungen erfüllt
  sind: 
  \begin{enumerate}[label=(\roman*)]
  \item Die Bilinearform \(a(\cdot, \cdot)\) ist \(Z\)-elliptisch:
    \begin{equation}\label{eq:5}
      \exists \alpha > 0: \forall z \in Z: a(z,z) \geq \alpha \vnorm{z}^2.
    \end{equation}

  \item Die Bilinearform \(b(\cdot, \cdot)\) erfüllt die
    inf-sup-Bedingung: 
    \begin{equation}
      \label{eq:6}
      \exists \beta > 0: \inf_{q\in Q} \sup_{v\in V} \frac{b(v,q)}{\vnorm{v}\norm{q}_Q} \geq \beta.
    \end{equation}
  \end{enumerate}
\end{theorem}

\noindent Der Beweis dieses Satzes nutzt das \emph{closed range theorem} und
wird hier nicht angegeben. \\
Wir bemerken, dass wir statt der Elliptizität der Bilinearform \(b(\cdot,
\cdot)\) die inf-sup-Bedingung für Satz \ref{thm:1} fordern
müssen. Desweiteren weisen wir darauf hin, dass die Elliptizität von
\(a(\cdot, \cdot)\) nur auf dem Unterraum \(Z\) gefordert wird. \\


\paragraph{Existenz von Lösungen für das Stokes Problem}
\label{sec:exist-von-losung}

Zum Beweis der Existenz und Eindeutigkeit von Lösungen für das Stokes
Problem \ref{problem:1} müssen wir also nach Theorem \ref{thm:1} die
\(Z\)-Elliptizität von \(a(\cdot, \cdot)\) und die inf-sup-Bedingung
für \(b(\cdot, \cdot)\) nachweisen. \\





{\color{red}
  \begin{enumerate}
  \item Normen von \(V\) und \(Q\) angeben?
  \item Definition von \(A:B\) angeben?
  \item Was genau bedeuted die inf-sup-Bedingung?
  \end{enumerate}

}

% In diesem Abschnitt soll zunächst eine Beschreibung des Stokes Problems
% gegeben werden. \\

% \noindent Dazu beginnen wir mit der starken Formulierung des Problems
% und geben anschließend eine gemischte Variationsformulierung an, die
% dann später auch als Grundlage für die Diskretisierung mit Hilfe des
% Crouzeix-Raviart Elements dienen soll.  


% \subsection{Starke Formulierung des Stokes Problems}
% \label{sec:starke-form-der}

% Die Stokesschen Gleichungen beschreiben die Bewegungen einer
% inkompressiblen Flüssigkeit in einem Körper. \\

% \noindent Als Modellproblem betrachten wir das Stokes Problem mit
% homogenen Dirichletbedingungen auf einem polygonalem
% Lipschitz-Gebiet. 

% \begin{modellproblem} \label{problem:1}
%   Sei \(\Omega \subset \real^2\) ein polygonales Lipschitz-Gebiet und
%   \(\fbold \in C^0(\Omega; \real^2)\). \\
%   Finde \(\ubold \in C^2(\Omega; \real^2) \cap C^0(\bar{\Omega}; \real^2)\)
%   und \(p \in C^1(\Omega)\) so dass gilt: 
%   \begin{equation} 
%     \label{eq:1}
%     \begin{cases}
%       - \Delta \ubold + \nabla p = \fbold  & \text{ in } \Omega, \\
%       \divOp \ubold = 0 & \text{ in } \Omega, \\
%       \ubold = 0 & \text{ auf } \partial \Omega. 
%     \end{cases}
%   \end{equation}
%   Da durch diese Angaben der Druck \(p\) nur bis auf eine Konstante
%   festgelegt wird, fordern wir zusätzlich
%   \[\int_\Omega p(x) \dx = 0.\]\\

%   \noindent Dabei beschreibt \(\ubold\) den Geschwindigkeitsvektor der
%   Flüssigkeit und \(p\) den Druck. Die Funktion \(\fbold\) ist die gegebene
%   äußere Kraftdichte. 
% \end{modellproblem}


% \subsection{Gemischte Variationsformulierung des Stokes Problems}
% \label{sec:gemischte-vari-des}

% Durch Multiplikation mit geeigneten Testfunktionen und Integration
% über das Gebiet \(\Omega\) erhält man die schwache Formulierung des
% Modellproblems \ref{problem:1}.  

% \begin{modellproblem} \label{problem:2}
%   Sei \(\Omega \subset \real^2\) ein polygonales Lipschitz-Gebiet und
%   \(f\in \Ltwo(\Omega; \real^2)\). Wir
%   definieren die Räume
%   \[V \coloneqq \Hzero(\Omega; \real^2) \text{ und } Q \coloneqq
%   \Ltwo_0(\Omega) \coloneqq \{q \in \Ltwo(\Omega): \int_\Omega q(x)
%   \dx = 0\},\]
%   sowie die Bilinearformen \(a: V\times V \rightarrow \real\)
%   und \(b: V\times Q \rightarrow \real\) durch
%   \begin{equation}
%     \label{eq:2}
%     a(u,v) \coloneqq \int_\Omega \nabla u \colon \nabla v \dx 
%     \text{ und }
%     b(v,q) \coloneqq \int_\Omega \divOp v \cdot q \dx. 
%   \end{equation}
%   Finde \((u,p) \in V\times Q\), sodass folgende Gleichungen erfüllt
%   sind: 
%   \begin{equation}
%     \label{eq:3}
%     \begin{cases}
%       a(u,v) - b(v,p) = \langle f, v \rangle_{\Ltwo(\Omega;\real^2)} &
%       \text{ für alle } v \in V \\
%       b(u,q) = 0 & \text{ für alle } q \in Q.
%     \end{cases}
%   \end{equation}
% \end{modellproblem}

% \noindent Ein Problem von der Form (\ref{eq:3}) bezeichnet man auch als
% Sattelpunktproblem. \textcolor{red}{Warum??}

% \noindent Die Existenz und Eindeutigkeit der Lösung \((u,p)\) lässt
% sich mit Hilfe des Splitting-Theorems von Brezzi zeigen. 


\end{document}
